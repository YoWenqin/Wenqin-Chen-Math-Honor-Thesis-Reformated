% Chapter Template

\chapter{Postulates of Quantum Mechanics} % Main chapter title

\label{Chapter3-postulates} % Change X to a consecutive number; for referencing this chapter elsewhere, use \ref{ChapterX}

\begin{quote}
\textit{"[T]he atoms or elementary particles themselves are not real; they form a world of potentialities or possibilities rather than one of things or facts."}
\bigskip
\hfill \textit{--Werner Heisenberg}
\end{quote}

In the 20th century, quantum mechanics fundamentally changed our understanding of the physical world.  Previously, physicists viewed the universe deterministically.  Any physical attribute of an object could be known, in principle, to an arbitrary degree of accuracy.  An object moving through space possessed an exact location, velocity and momentum at a time $t$. In contrast, in the quantum mechanical universe Schrödinger's famous cat is simultaneously alive and dead subject to a certain probability distribution.  Only when we actually opens the chamber to observe the cat is its ultimate state determined. If an object is moving, the certainty with which we know its position constrains how much we can know about its velocity.  Perhaps even more alarmingly, an observer cannot even measure an object without changing it.

Famous physicists like Niels Bohr, Max Planck, Paul Dirac, Werner Heisenberg, Richard Feynman and many others relegated the previously held {\emph{classical}} view of the universe to only one side of a coin.  In this chapter we present the postulates of quantum mechanics.  In each instance, after stating a postulate precisely, we summarize the key ideas, making explicit what the reader should take from the postulate for this thesis.  This summary is by no means complete.  See \cite{hall2013quantum} for a more comprehensive introduction to quantum mechanics for mathematicians, or \cite{rieffel1998introduction} for a computer science perspective. In addition, \cite{dorai2018} provides a summary in the history of quantum mechanics.

%----------------------------------------------------------------------------------------
%	SECTION 1
%----------------------------------------------------------------------------------------
\pagebreak
\section{Postulate 1: State space}


\begin{quote}
    \textbf{Postulate 1}: Associated to any isolated physical system is a Hilbert space\footnote{a complete inner product space} known as the {\emph{state space}} of the system. The system is completely described by its {\emph{state vector}}, which is a unit vector in the system's state space.
\end{quote}

For our purposes, the Hilbert space is $C^n$. In fact, often $n=2$, so the state vector is a qubit (see \textbf{Definition \ref{def qubit}}).  Thus, in many of our examples, a state vector is a superposition of $\ket{0}$ and $\ket{1}$.

There are {\emph{real}} physical systems that can be described in terms of qubits.  These include, for example, the polarization of a photon, the alignment of nuclear spin in a uniform magnetic field, and the state of an electron orbiting a single atom.  In the last case, the orbiting electron can exist in either the "ground" or "excited" states, which can be identified with $\ket{0}$ and $\ket{1}$ respectively. By shining light on the atom for an appropriate length of time, it is possible to move the electron from the $\ket{0}$ state to the $\ket{1}$. Interestingly, by reducing the time we shine the light, we may move the electron from the state $\ket{0}$ to $\ket{+}$, a (nontrivial) superposition
\footnote{
$\ket{+}$ and $\ket{-}$ can both be viewed as being "halfway" between $\ket{0}$ and $\ket{1}$} of $\ket{0}$ and $\ket{1}$ (see \cite{Nielsen}).  In fact, the sates $\ket{+}$ and $\ket{-}$ form a commonly used orthonormal basis in $\mathbb{C}^2$ known as the \textit{Hadamard basis}. They are defined as follows:
\begin{equation} \label{eqn: hadamard basis}
\ket{+}=\frac{1}{\sqrt{2}}(\ket{0}+\ket{1}),\hspace{.2 cm}
\ket{-}=\frac{1}{\sqrt{2}}(\ket{0}-\ket{1}).
\end{equation}


% \textcolor{blue}{***Raj: It might be good to define $\ket{+}$ and $\ket{-}$ earlier. You use them again in the next postulate.***}
\pagebreak
%----------------------------------------------------------------------------------------
%	SECTION 2
%----------------------------------------------------------------------------------------

\section{Postulate 2: Evolution}

\begin{quote}
    \textbf{Postulate 2}: The evolution of a closed quantum system is described by a unitary transformation. More precisely, the state $\ket{\psi_1}$ of the system at time $t_1$ is related to the state $\ket{\psi_2}$ of the system at time $t_2$ by a unitary operator U.  So, $\ket{\psi_2}=U\ket{\psi_1}$, where $U$ depends only on the times $t_1$ and $t_2$.
\end{quote}

Since for us, the state space will be ${\mathbb{C}}^n$ or ${\mathbb{C}}^2$, our operators will all be unitary matrices.  

\begin{example} [Hadamard gate]
A commonly used unitary operator is the Hadarmard gate, denoted by H.  
\begin{equation}
   H=\frac{1}{\sqrt{2}}\begin{pmatrix}
1 && 1\\
1 && -1
\end{pmatrix} 
\end{equation}
Clearly $H$ satisfies,
$$H\ket{0}=\ket{+}, \textrm{ and }H\ket{1}=\ket{-},$$
so $H$ maps an orthonormal set to an orthonormal set as we have seen.

\end{example}

% \begin{example} [Pauli Matrices]

% \end{example}
Note that this Postulate does not tell us what unitary operator describes the evolution of the system, it only assures that the evolution can be described in such way.  When the state space is ${\mathbb{C}}^2$, any unitary operator will show up in such a way.

It's worth pointing out that {\bf{Postulate 2}} applies to a \emph{closed} system.  While this is nearly impossible in reality, in many of the scenarios discussed in this thesis it is reasonable to describe the evolution via unitary operators nonetheless.





%----------------------------------------------------------------------------------------
%	SECTION 3
%----------------------------------------------------------------------------------------
\pagebreak
\section{Postulate 3: Quantum Measurement} \label{section: postulate 3}
%-----------------------------------
%	SUBSECTION 1
%-----------------------------------



\begin{quote}
    \textbf{Postulate 3}: Quantum measurements are described by a collection $\{M_m\}$ of measurement operators which satisfy the {\emph{completeness equation}} $\sum\limits_m M_m^\dagger M_m =\mathbb{I}$. These operators act on the state space of the system being measured, and the index {\emph{m}} refers to the measurement outcomes that may occur in the experiment. If the quantum system is in state $\ket{\psi}$ immediately prior to measurement, then the probability that outcome {\emph{m}} occurs is given by 
    \begin{equation} \label{eqn: general measurement probability}
        p(m)=\bra{\psi}M_m^\dagger M_m \ket{\psi}.
    \end{equation} 
   When this is the case, the state of the system after the measurement is
    \begin{equation} \label{eqn: general measurement post state}
        \frac{M_m \ket{\psi}}{\sqrt{\bra{\psi}M_m^\dagger M_m \ket{\psi}}}.
    \end{equation}

\end{quote}




This postulate is crucial going forward, so we will provide lots of examples, and will spend some time of the postulate.  We should point out that when {\emph{any}} external object (including measuring equipment) observes a quantum system, the quantum system is no longer closed, and hence is not necessarily subject to unitary evolution as in {\bf{Postulate 2}}. {\bf{Postulate 3}} describes the effects of these measurements on a quantum system. 

We start with a simple but important example.
\begin{example} \label{measurement-standard basis}
We measure a single qubit with two outcomes given by the two measurement operators 
$$M_0=\ket{0}\bra{0}\textrm{ and }M_1=\ket{1}\bra{1}.$$ 
Each measurement operator is projector, and thus, by our previous computation,
\begin{equation*}
    M_0^\dagger M_0+M_1^\dagger M_1 = (M_0)^2+(M_1)^2=M_0 + M_1 = \mathbb{I}.
\end{equation*}
Now, suppose the state being measured is $\ket{\psi}=a\ket{0}+b\ket{1}$. Then the probability of obtaining measurement outcome 0 is
\begin{equation*}
    p(0) = \bra{\psi}M_0^\dagger M_0 \ket{\psi} = \braket{\psi | M_0 | \psi}=\braket{\psi|0}\braket{0|\psi}=\braket{\psi|o}\overline{\braket{\psi|0}}=|a|^2.
\end{equation*}
Similarly, the probability of obtaining measurement outcome 1 is $|b|^2$.
The state after measurement in the two cases is therefore given by
\begin{gather*}
    \frac{M_0 \ket{\psi}}{\sqrt{|a|^2}}=\frac{\ket{0}\bra{0}\ket{\psi}}{|a|}=\frac{a}{|a|}\ket{0}\\
    \frac{M_1 \ket{\psi}}{\sqrt{|b|^2}}=\frac{\ket{1}\bra{1}\ket{\psi}}{|b|}=\frac{b}{|b|}\ket{1}\\   
\end{gather*}
Since $\ket{\psi}$ is a unit vector, $p(0)+p(1)=1$.  Thus, the measurement outcome is $0$ with probability $|a|$, in which case the post-measurement state is a unit vector in the direction of $\ket{0}$.  Similarly, the measurement outcome is $1$ with probability $|b|$, in which case the post-measurement state is a unit vector in the direction of $\ket{1}$.
\end{example}

If two quantum states $\ket{\psi}, \ket{\sigma}$ satisfy $\ket{\psi}=e^{i\theta}\ket{\sigma}$, we say $\ket{\psi}$ is equal to $\ket{\sigma}$ up to the \textit{global phase factor $e^{i\theta}$}. For quantum mechanical purposes, these states are equal because the probability distribution corresponding to either outcome is the same. This is because if $M_m$ is a measurement operator used in an arbitrary quantum measurement, then the probability of observing outcome $m$ for $\ket{\psi}$ is 
\begin{eqnarray}
\braket{\psi|M_m^\dagger M_m|\psi} &=& \braket{e^{i\theta}\sigma|M_m^\dagger M_m|e^{i\theta} \sigma} \\
&=& |e^{i\theta}|^2\braket{\sigma|M_m^\dagger M_m| \sigma} \\
&=& 1^2\braket{\sigma|M_m^\dagger M_m| \sigma} \\
&=& \braket{\sigma|M_m^\dagger M_m| \sigma} ,
\end{eqnarray}
which equals the probability of observing outcome $m$ for $\ket{\sigma}$.  Therefore, the global phase factors $\frac{a}{|a|}, \frac{b}{|b|}$ in \textbf{Example \ref{measurement-standard basis}} are irrelevant from an observational point of view.
% \begin{prop}
% $\ket{\psi}\bra{\psi}=\ket{\sigma}\bra{\sigma} =>$ there exists an angle $\theta$ such that $\ket{\psi}=e^{i\theta}\ket{\sigma}$.
% \textcolor{red}{what's this proposition for?}
% \end{prop}

\bigskip
It follows from {\bf{Postulate 3}} that {\emph{only orthogonal states}} can be reliably distinguished. To illustrate this property, we consider two situations where Alice and Bob have access to a known collection of states.  In both situations, Alice chooses a state and we examine whether Bob can reliably determine which state Alice picked by measuring it.  Consider first the following example.

\begin{example} \label{example: orthogonal states for measurements}
Suppose Alice chooses a state $\ket{\psi_i}$ from some fixed set of states known to both herself and to Bob. She gives the state to Bob, whose task is to identify the index $i$ from the state Alice has given him by measuring it. If the collection of states $\{\ket{\psi_j}\}_{j=1}^n$ is orthonormal, then Bob can determine the index i correctly with certainty.
\end{example}
\begin{proof}
We describe the procedure for Bob to conduct his quantum measurement.  Define measurement operators $M_j=\ket{\psi_j}\bra{\psi_j}$ for each $j$.  We have seen that each $M_j$ is a Hermitian projector, and that an orthonormal basis satisfies the completeness relation.  The key point is that Bob may measure with {\emph{this collection of measurement operators.}}

Thus, if Alice picked $\ket{\psi_i}$, the probability of observing $i$ and $j\neq i$ respectively are given by;
\begin{align}
p(i)=\braket{\psi_i | M_i |\psi_i}=\braket{\psi_i |\psi_i}\braket{\psi_i |\psi_i}=1^2=1\\
p(j)=\braket{\psi_i | M_j |\psi_i}=\braket{\psi_i |\psi_j}\braket{\psi_j |\psi_i}=0^2=0.
\end{align}
This computation shows that Bob can detect the outcome $i$ with probability one.  Once again the crucial point is that the set of vectors from which Alice selects is known to Bob and is {\emph{orthonormal}}. Because of this, he can measure with respect to the corresponding collection of measurement operators. 
\end{proof}

On the other hand, {\emph{even when they are known to Bob,}} non-orthogonal states cannot be reliably distinguished (see page 87 Box 2.3 in \cite{Nielsen} for a rigoroous proof).  We demonstrate this phenomenon with another example.
\begin{example} \label{example: non-orthogonal states for measurements}
This time, say Alice selects from the states $\ket{\psi_1}=\ket{0}$ and  $\ket{\psi_2}=\ket{+}$. Suppose Bob attempts to distinguish $\ket{\psi_1}, \ket{\psi_2}$ with measurement operators $M_1=\ket{0}\bra{0}, M_2=\ket{1}\bra{1}$ as before.

Now, if Alice picks $\ket{\psi_1}$, then $p(1)=\braket{0 |0}\braket{0|0}=1$, so Bob picks the correct index. 

However, if Alice picks $\ket{\psi_2}$, then the probabilities of the measurement outcomes being $1$ and $2$ respectively are given by;
\begin{eqnarray}
    & &p(1)=\braket{\psi_2|0}\braket{0|\psi_2}=|\braket{0|\psi_2}|^2=|\frac{\braket{0|0}+\braket{0|1}}{\sqrt{2}}|^2=\frac{1}{2}\textrm{  and}\\
    & &p(2)=\braket{\psi_2|1}\braket{1|\psi_2}={\frac{1}{2}}_.
\end{eqnarray}
This means is that if Bob's measurement outcome is $2$, Alice must have picked $\ket{\psi_2}$, so Bob will be able to guess correctly with certainty. However, if the measurement outcome is $1$, Alice could have prepared either $\ket{\psi_1}$ or $\ket{\psi_2}$.  Thus, Bob will not be able to determine Alice's index with certainty.  In fact, suppose that Alice prepares $\ket{\psi_1}$ and $\ket{\psi_2}$ with equal probability.  Then if Bob guesses that Alice has prepared $\ket{\psi_1}$, the probability of him being correct is $\frac{\frac{1}{2}}{\frac{1}{2}+\frac{1}{2}\frac{1}{2}}=\frac{2}{3}$.
\end{example}

%-----------------------------------
%	SUBSECTION 2
%-----------------------------------

\label{subsection:projective measurement}
The measurement operators used in the last {\bf{Examples}} \ref{example: orthogonal states for measurements}, \ref{example: non-orthogonal states for measurements} are by far the most important kind for the purposes of this thesis.  Such measurements are known as \textit{projective measurements}, and they are \textit{equivalent} to the general measurements, when coupled with unitary transformations as described in {\bf{Postulate 2}} \footnote{See \cite{Nielsen} Section 2.2.8 for why they are equivalent}. 


Note that whenever the measurement operators in \textbf{Postulate 3}, are orthogonal projectors the completeness condition becomes
\begin{equation}
\sum\limits_m M_m^\dagger M_m = \mathbb{I} = \sum\limits_m M_m M_m = \sum\limits_m M_m.
\end{equation}

% \textcolor{blue}{Insert definition of projective measurement with respect to an orthonormal basis here (begun below).  Add a footnote saying "We will define projective measurements rigorously in the Appendix, but for our purposes projective measurements can be made with respect to an orthonormal basis."  I think there is no need to include the more general definition of projective measurement except in an appendix} \textcolor{green}{done}

We now define the type of measurement we will use from this point forward.  

\begin{definition} \label{def: measurement in a basis state vector}
Let $\{\ket{u_m}\}$, be an orthnormal basis for $\mathbb{C}^n$.  When we measure with respect to the family of operators $\{M_m\}=\{\ket{u_m}\bra{u_m}\}$, we say that we are measuring with respect to the orthonormal basis $\{\ket{u_m}\}$\footnote{These measurements are a special type of \textit{projective measurement}.  We define projective measurements rigorously in the \textbf{Appendix \ref{AppendixB}}.}.
\end{definition}
Note that if the quantum system is in state $\ket{\psi}$ immediately prior to measurement, then the probability of that outcome $m$ occurs is given by
\begin{equation}\label{eqn: basis measurement probability}
    p(m)=\braket{\psi|M_m|\psi}=\braket{\psi\ket{u_m}\bra{u_m}\psi}   =|\braket{\psi|u_m}|^2.
\end{equation}

One easily checks that each $ket{u_m}\bra{u_m}$ is a Hermitian projector.  By making use of these properties, we see that when the outcome $m$ is observed, the state of the system after post-measurement is
\begin{equation} 
\frac{M_m \ket{\psi}}{\sqrt{\bra{\psi}M_m^\dagger M_m \ket{\psi}}}=\frac{\ket{u_m}\bra{u_m} \ket{\psi}}{\sqrt{\bra{\psi}\ket{u_m}\bra{u_m} \ket{\psi}}}=\frac{\braket{u_m|\psi}\ket{u_m}}{|\braket{\psi|u_m}|}=\ket{u_m}.
\end{equation}



By the computation in \textbf{Section \ref{section:complex vector space}}, we see that $\{\ket{e_m}\bra{e_m}\}$ is a family of operators as defined in \textbf{Postulate 3}.  The calculation in \textbf{Example \ref{measurement-standard basis}} is exactly this computation, applied to the standard basis in  $\mathbb{C}^2$.  Here is an example using the Hadamard basis.

% \textcolor{blue}{Note that .. (refer to above examples).  Here is an additional example ... Include example of a projectvie measurement with respect to an orthonorma basis that isn't the standard basis}

\begin{example}
Suppose the initial quantum state is $\ket{\psi}=\ket{0}$. We measure in the Hadamard basis $\{\ket{+}, \ket{-}\}$. Then by \eqref{eqn: basis measurement probability}, the probabilities that outcome 1, 2 occurs are each given by
\begin{gather}
    p(1)=|\braket{0|+}|^2=\frac{1}{2}\\
    p(2)=|\braket{0|-}|^2=\frac{1}{2}
\end{gather}
where the state of the system after the measurement is $\ket{+}, \ket{-}$ respectively.
\end{example}



% For our purposes the most important measurements are projective measurements with respect to an orthonormal basis, as in \textbf{Examples} \ref{measurement-standard basis} and \ref{example: orthogonal states for measurements}.  
We again emphasize that in this paper, when say we are "measuring" we are always measuring with respect to an orthonormal basis.


% Note that when we measure the quantum state $\ket{\psi}$ with respect to the orthonormal basis $B =\{\ket{b_j}\}_{j=1}^d$, the probability of observing outcome $j$ is given by,
%  \begin{equation}
%  p(j)=\braket{\psi|b_j}\braket{b_j|\psi}= \overline{\braket{b_j|\psi}}\braket{b_j|\psi}= |\braket{b_j|\psi}|^2.
%  \end{equation}
%  When this outcome is observed, the post-measurement state is $\ket{b_j}$, up to a global phase.  Thus, \textbf{Example} \ref{measurement-standard basis} is exactly this computation applied to the orthonormal basis $\{\ket{0},\ket{1}\}.$
 \pagebreak
 
 
 
 
%  ************************************************************
 
%  \textcolor{blue}{BELOW THIS LINE MOVE TO APPENDIX}


%----------------------------------------------------------------------------------------
%	SECTION 4
%----------------------------------------------------------------------------------------
\pagebreak
\section{Postulate 4: Composite Systems} \label{section: composite systems}
\begin{quote}
    \textbf{Postulate 4}:  The state space of a composite physical system is the tensor product of the component physical systems. Moreover, if we have $n$ systems numbered $1$ through $n$, where system $i$ is in the state $\ket{\psi_i}$, then the joint state of the total system is $\ket{\psi_1} \otimes \ket{\psi_1} \otimes \hdots \otimes \ket{\psi_n}$
\end{quote}



{\bf{Postulate 4}} makes clear the crucial role tensor products play in quantum mechanics.  This Postulate also enables us to define one of the most important ideas associated with composite quantum systems, that of \textit{entanglement}. 


\begin{definition}\label{definition: entanglement with state vector}
We say that a state of a composite system is \textit{separable} if it is the tensor product of two vectors in its component systems. Otherwise, we say the composite system is \textit{entangled}.
\end{definition}


\begin{example} \label{example: entangled state}
Consider the two qubit state 
\begin{equation}\label{eqn: epr}
    \ket{EPR}=\frac{\ket{00}+\ket{11}}{\sqrt{2}}.
\end{equation}

There are no single qubit states $\ket{a}, \ket{b}$ such that $\ket{\psi}=\ket{a}\otimes \ket{b}$, hence $\ket{\psi}$ is entangled.
\begin{proof}
Suppose for a contradiction, $\psi = \ket{a} \otimes \ket{b}.$  Then, 
$$\ket{a}=\alpha \ket{0}+\beta \ket{1}, \ket{b}=\gamma\ket{0}+\delta\ket{1}.$$
Then,
\begin{eqnarray}
\ket{a}\otimes \ket{b} &=& (\alpha \ket{0}+\beta \ket{1}) \otimes (\gamma\ket{0}+\delta\ket{1})\\
&=& \alpha \gamma \ket{00}+\alpha \delta \ket{01}+\beta \gamma\ket{10}+\beta\delta\ket{11}\\
&=&\frac{1}{\sqrt{2}}\ket{00}+0\ket{01}+0\ket{10}+\frac{1}{\sqrt{2}}\ket{11}\\
&=&\ket{\psi}.
\end{eqnarray}
Therefore $\alpha \gamma \neq 0$ and $\beta \delta \neq 0$, so all are nonzero.  But $\alpha \delta =0$ (and $\beta \gamma=0$), a contradiction.
\end{proof}
\end{example}


Informally, when two states are maximally entangled, one cannot be described physically independent of the other.  In practice, we may think of entangled states as being {\emph{correlated}} in the sense that information about one of them is in fact, information about {\emph{both}} of them. In {\bf{Chapter}} \ref{Chapter6-classification of entanglement} we will classify entanglement completely and examine more precisely the way in which entangled states are correlated. 

Entangled states adhere to {\emph{monogamy of entanglement}} (see \cite{Toner2006}).  This principal is used in cryptography protocols such as E91 which we will discuss in \textbf{Section \ref{section: e91}}. Briefly, monogamy states that if two qubits A and B are maximally entangled, they cannot be entangled with a third qubit C. 


