% Chapter Template

\chapter{Postulates of Quantum Mechanics} % Main chapter title

\label{Chapter3-postulates} % Change X to a consecutive number; for referencing this chapter elsewhere, use \ref{ChapterX}

\begin{quote}
\textit{"[T]he atoms or elementary particles themselves are not real; they form a world of potentialities or possibilities rather than one of things or facts."}
\bigskip
\hfill \textit{--Werner Heisenberg}
\end{quote}

In the 20th century, quantum mechanics fundamentally changed our understanding of the physical world.  Previously, physicists viewed the universe deterministically.  Any physical attribute of an object could be known, in principle, to an arbitrary degree of accuracy.  An object moving through space possessed an exact location, velocity and momentum at a time $t$. In contrast, in the quantum mechanical universe Schrödinger's famous cat is simultaneously alive and dead subject to a certain probability distribution.  Only when we actually opens the chamber to see whether the cat is alive or dead is it's ultimate state determined. If an object is moving, the certainty wit which we know its position, constrains how much we can know about its velocity.  Perhaps even more alarmingly, an observer cannot even measure an object without changing it.

Famous physicists like Niels Bohr, Max Planck, Paul Dirac, Werner Heisenberg, Richard Feynman and many others relegated the previously held {\emph{classical}} view of the universe to only one side of a coin.  In this chapter we present the postulates of quantum mechanics.  In each instance, after stating a postulate precisely, we summarize the key ideas, making explicit what the reader should take from the postulate for this thesis.  This summary is by no means complete.  See ... \textcolor{red}{add more references that provide introductions to quantum mechanics for non-physicists} for a more extensive introduction to the subject and \cite{dorai2018} for a more extensive introduction to the history of quantum mechanics.

%----------------------------------------------------------------------------------------
%	SECTION 1
%----------------------------------------------------------------------------------------
\pagebreak
\section{Postulate 1: State space}


\begin{quote}
    \textbf{Postulate 1}: Associated to any isolated physical system is a Hilbert space\footnote{a complete inner product space} known as the state space of the system. The system is completely described by its state vector, which is a unit vector in the system's state space.
\end{quote}

For our purposes, typically the Hilbert space is $C^n$. In fact, most of the time $n=2$, so the state vector is a qubit (see \textbf{Definition \ref{def qubit}} and more generally Chapter \ref{Chapter2-preliminaries}).  Thus, unless otherwise specified, in what follows a state vector is a superposition of $\ket{0}$ and $\ket{1}$.

There are {\emph{real}} physical systems that can be described in terms of qubits.  These include, for example, the polarization of a photon, the alignment of nuclear spin in a uniform magnetic field, and the state of an electron orbiting a single atom.  In the last case, the orbiting electron can exist in either the "ground" or "excited" states, which can be identified with $\ket{0}$ and $\ket{1}$ respectively. By shining light on the atom for an appropriate length of time, it is possible to move the electron from the $\ket{0}$ state to the $\ket{1}$. Interestingly, by reducing the time we shine the light, we may move the electron from the state $\ket{0}$ to $\ket{+}$, a (nontrivial) superposition\footnote{$\ket{+}=\frac{\sqrt{2}}{2}\ket{0}+\frac{\sqrt{2}}{2}\ket{1}$, hence $\ket{+}$ can be viewed as being "halfway" between $\ket{0}$ and $\ket{1}$} of $\ket{0}$ and $\ket{1}$ (see \cite{Nielsen}).


%----------------------------------------------------------------------------------------
%	SECTION 2
%----------------------------------------------------------------------------------------
\pagebreak
\section{Postulate 2: Evolution}

\begin{quote}
    \textbf{Postulate 2}: The evolution of a closed quantum system is described by a unitary transformation. More precisely, the state $\ket{\psi_1}$ of the system at time $t_1$ is related to the state $\ket{\psi_2}$ of the system at time $t_2$ by a unitary operator U.  So, $\ket{\psi_2}=U\ket{\psi_1}$, where $U$ depends only on the times $t_1$ and $t_2$.
\end{quote}

Since for us, the state space will be ${\mathbb{C}}^n$ or ${\mathbb{C}}^2$, our operators wil all be unitary matrices.  

\begin{example} [Hadamard gate]
A commonly used unitary operator is the Hadarmard gate, denoted by H.  
\begin{equation}
   H=\frac{1}{\sqrt{2}}\begin{pmatrix}
1 && 1\\
1 && -1
\end{pmatrix} 
\end{equation}
Clearly $H$ satisfies,
$$H\ket{0}=\ket{+}, \textrm{ and }H\ket{1}=\ket{-},$$
so $H$ maps an orthonormal set to an orthonormal set as we have seen.

\end{example}

% \begin{example} [Pauli Matrices]

% \end{example}
Note that this postulate does not tell us what unitary operator describes the evolution of the system, it only assures that the evolution can be described in such way.  When the state space is ${\mathbb{C}}^2$, any unitary operator will show up in such a way.

It's worth pointing out that Postulate 2 applies to a {\emph{closed} system.  While this is nearly an impossible in reality, in many of the scenarios discussed in this thesis it is reasonable to describe the evolution via unitary operators nonetheless.





%----------------------------------------------------------------------------------------
%	SECTION 3
%----------------------------------------------------------------------------------------
\pagebreak
\section{Postulate 3: Quantum Measurement}
%-----------------------------------
%	SUBSECTION 1
%-----------------------------------



\begin{quote}
    \textbf{Postulate 3}: Quantum measurements are described by a collection $\{M_m\}$ of measurement operators. These operators act on the state space of the system being measured, and the index {\emph{m}} refers to the measurement outcomes that may occur in the experiment. If the quantum system is in state $\ket{\psi}$ immediately prior to measurement, then the probability that outcome {\emph{m}} occurs is given by 
    \begin{gather*}
        p(m)=\bra{\psi}M_m^\dagger M_m \ket{\psi},
    \end{gather*}
    in which case the state of the system after the measurement is
    \begin{gather*}
        \frac{M_m \ket{\psi}}{\sqrt{\bra{\psi}M_m^\dagger M_m \ket{\psi}}}.
    \end{gather*}
Lastly, the measurement operators are required to satisfy the completeness equation $\sum\limits_m M_m^\dagger M_m =I$.
\end{quote}


This postulate is of importance going forward, so we will provide lots of examples, and will spend some time of the Postulate.  We should point out that when {\emph{any}} external object (including measuring equipment) observes a quantum system, the quantum system is no longer closed, and hence is not necessarily subject to unitary evolution as in Postulate 2. Postulate 3 describes the effects of these measurements on a quantum system. 

We start with a simple but important example.
\begin{example} \label{measurement-standard basis}
We measure a single qubit with two outcomes given by the two measurement operators 
$$M_0=\ket{0}\bra{0}\textrm{ and }M_1=\ket{1}\bra{1}.$$ 
Each measurement operator is projector, and thus, by our previous computation,
\begin{equation*}
    M_0^\dagger M_0+M_1^\dagger M_1 = (M_0)^2+(M_1)^2=M_0 + M_1 = I.
\end{equation*}
Now, suppose the state being measured is $\ket{\psi}=a\ket{0}+b\ket{1}$. Then the probability of obtaining measurement outcome 0 is
\begin{equation*}
    p(0) = \bra{\psi}M_0^\dagger M_0 \ket{\psi} = \braket{\psi | M_0 | \psi}=\braket{\psi|0}\braket{0|\psi}=\braket{\psi|o}\overline{\braket{\psi|0}}=|a|^2.
\end{equation*}
Similarly, the probability of obtaining measurement outcome 1 is $|b|^2$.
The state after measurement in the two cases is therefore given by
\begin{gather*}
    \frac{M_0 \ket{\psi}}{\sqrt{|a|^2}}=\frac{\ket{0}\bra{0}\ket{\psi}}{|a|}=\frac{a}{|a|}\ket{0}\\
    \frac{M_1 \ket{\psi}}{\sqrt{|b|^2}}=\frac{\ket{1}\bra{1}\ket{\psi}}{|b|}=\frac{b}{|b|}\ket{1}\\   
\end{gather*}
Since $\ket{\psi}$ is a unit vector, $p(0)+p(1)=1$.  Thus, the measurement outcome is $0$ with probability $|a|$, in which case the post-measurement state is a unit vector in the direction of $\ket{0}$.  Similarly, the measurement outcome is $1$ with probability $|b|$, in which case the post-measurement state is a unit vector in the direction of $\ket{1}$.
\end{example}

\textcolor{blue}{INCLUDE SCHEMATIC HERE describing the outcome, probability, direction}



If two quantum states $\ket{\psi}, \ket{\sigma}$ satisfy $\ket{\psi}=e^{i\theta}\ket{\sigma}$, we say $\ket{\psi}$ is equal to $\ket{\sigma}$ up to the \textit{global phase factor $e^{i\theta}$}. For quantum mechanical purposes, they are considered equal because the probability distribution corresponding to either outcome is the same. This is because if $M_m$ is a measurement operator used in an arbitrary quantum measurement, then the probability of observing outcome $m$ for $\ket{\psi}$ is 
\begin{eqnarray}
\braket{\psi|M_m^\dagger M_m|\psi} &=& \braket{e^{i\theta}\sigma|M_m^\dagger M_m|e^{i\theta} \sigma} \\
&=& |e^{i\theta}|^2\braket{\sigma|M_m^\dagger M_m| \sigma} \\
&=& 1^2\braket{\sigma|M_m^\dagger M_m| \sigma} \\
&=& \braket{\sigma|M_m^\dagger M_m| \sigma} ,
\end{eqnarray}
which equals the probability of observing outcome $m$ for $\ket{\sigma}$.  Therefore, the global phase factors $\frac{a}{|a|}, \frac{b}{|b|}$ in Example \ref{measurement-standard basis} are irrelevant from an observational point of view.
% \begin{prop}
% $\ket{\psi}\bra{\psi}=\ket{\sigma}\bra{\sigma} =>$ there exists an angle $\theta$ such that $\ket{\psi}=e^{i\theta}\ket{\sigma}$.
% \textcolor{red}{what's this proposition for?}
% \end{prop}

\bigskip
It follows from Postulate 3 that only orthogonal states can be reliably distinguished. To illustrate this property, we consider two situation where Alice and Bob have access to a known collection of states.  In both situations, Alice chooses a state and we examine whether Bob can reliably determine which state Alice picked by measuring it.

\begin{example} \label{example: orthogonal states for measurements}
Orthogonal states can be reliably distinguished. More precisely, suppose Alice chooses a state $\ket{\psi_i}$ from some fixed set of states known to both herself and to Bob. She gives the state to Bob, whose task is to identify the index $i$ from the state Alice has given him by measuring. If the collection of states $\{\ket{\psi_j}\}_{j=1}^n$ is orthonormal, then Bob can determine the index i correctly with certainty.
\end{example}
\begin{proof}
We describe the procedure for Bob to conduct his quantum measurement.  Define measurement operators $M_j=\ket{\psi_j}\bra{\psi_j}$ for each $j$.  We have seen that each $M_j$ is a Hermitian projector, and that an orthonormal basis satisfies the completeness relation.  The key point is that Bob may measure with {\emph{this collection of measurement operators.}}

Thus, if Alice picked $\ket{\psi_i}$, the probability of observing $i$ and $j\neq i$ is given by;
\begin{align}
p(i)=\braket{\psi_i | M_i |\psi_i}=\braket{\psi_i |\psi_i}\braket{\psi_i |\psi_i}=1^2=1\\
p(j)=\braket{\psi_i | M_j |\psi_i}=\braket{\psi_i |\psi_j}\braket{\psi_j |\psi_i}=0^2=0\\
\end{align}
Thus, Bob can detect the outcome $i$ with probability $1$.  Once again the crucial point is that the set of vectors from which Alice selects is known to Bob and is {\emph{orthonormal}}. Because of this, he can measure with respect to the {\emph{appropriate collection of measurement operators.}} 
\end{proof}

On the other hand, {\emph{even when they are known to Bob,}} non-orthogonal states cannot be reliably distinguished (see page 87 Box 2.3 in \cite{Nielsen} for a rigoroous proof).  We demonstrate this phenomenon with another example.
\begin{example} \label{example: non-orthogonal states for measurements}
This time, say Alice selects from the states $\ket{\psi_1}=\ket{0}$ and  $\ket{\psi_2}=\frac{\ket{0}+\ket{1}}{\sqrt{2}}$. Suppose Bob attempts to distinguish $\ket{\psi_1}, \ket{\psi_2}$ with measurement operators $M_1=\ket{0}\bra{0}, M_2=\ket{1}\bra{1}$ as before.

Now, if Alice prepares $\ket{\psi_1}$, then $p(1)=\braket{0 |0}\braket{0|0}=1$, so Bob picks the correct index. 

However, if Alice prepares $\ket{\psi_2}$, then the probabilities of the measurement outcomes being $1$ and $2$ respectively are given by;
\begin{eqnarray}
    & &p(1)=\braket{\psi_2|0}\braket{0|\psi_2}=|\braket{0|\psi_2}|^2=|\frac{\braket{0|0}+\braket{0|1}}{\sqrt{2}}|^2=\frac{1}{2}\textrm{  and}\\
    & &p(2)=\braket{\psi_2|1}\braket{1|\psi_2}={\frac{1}{2}}_.
\end{eqnarray}
This means is that if Bob's measurement outcome is $2$, Alice must have prepared $\ket{\psi_2}$, so Bob will be able to guess correctly with certainty. However, if the measurement outcome is $1$, Alice could have prepared either $\ket{\psi_1}$ or $\ket{\psi_2}$.  Thus, Bob will not be able to determine Alice's index with certainty.  In fact, suppose that Alice prepares $\ket{\psi_1}$ and $\ket{\psi_2}$ with equal probability.  Then if Bob guesses that Alice has prepared $\ket{\psi_1}$, the probability of him being correct is $\frac{\frac{1}{2}}{\frac{1}{2}+\frac{1}{2}\frac{1}{2}}=\frac{2}{3}$.
\end{example}

%-----------------------------------
%	SUBSECTION 2
%-----------------------------------

\label{subsection:projective measurement}
The measurement operators used in the last two examples are by far the most important kind for the purposes of this thesis.  Such measurements are known as \textit{projective measurements}, and they will turn out to be \textit{equivalent} to the general measurements, when coupled with unitary transformations as described in Postulate 2 \footnote{See \cite{Nielsen} Section 2.2.8 for why they are equivalent}. 


To perform projective measurements, the measurement operators in \textbf{Postulate 3}, must be orthogonal projectors, in which case the completeness condition becomes
\begin{equation}
\sum\limits_m M_m^\dagger M_m = I = \sum\limits_m M_m M_m = \sum\limits_m M_m.
\end{equation}

We will define projective measurements rigorously in the Appendix, but for our purposes projective measurements can be made with respect to an orthonormal basis.

\begin{definition}
Let
\end{definition}

Postulate 3 can be completely restated in the context of projective measurements, which we present in the Appendix.

**************************************************


The measurement postulate for projective measurements is superficially rather different from the general postulate above.
\begin{quote}
    \textbf{Projective Measurements}: A projective measurement is described by an \textit{observable}, M, a Hermitian operator on the state space of the sytem being observed. The observable has a spectral decomposition,
    \begin{equation}
        M=\sum_m m P_m
    \end{equation}
    where $P_m$ is the projector onto the eigenspace of M with eigenvalue m \footnote{The eigenspace corresponding to an eigenvalue m is the set of vectors which have eigenvalue m}. The possible outcomes of the measurement correspond to the eigenvalues, m, of the observable. Upon measuring the state $\ket{\psi}$, the probability of getting result m is given by 
    \begin{equation}
        p(m)=\braket{\psi|P_m|\psi}
    \end{equation}
    Given that outcome m occurred, the state of the quantum system immediately after the measurement is 
    \begin{equation}
        \frac{P_m \ket{\psi}}{\sqrt{p(m)}}.
    \end{equation}
\end{quote}

Here is an example of projective measurements on single qubits.
\begin{example}
Consider the Pauli-Z matrix introduced in \textbf{Example} \ref{example-pauli z diagonal rep}.
\begin{equation}
    Z=\ket{0}\bra{0}-\ket{1}\bra{1}
\end{equation}
Take Z as the observable for out projective measurement. Here $P_{+1}=\ket{0}\bra{0}$ is the projector onto the eigenspace of Z with eigenvalue 1. $P_{-1}=\ket{1}\bra{1}$ is the projector onto the eigenspace of Z with eigenvalue -1.

Thus the measurement of Z on the state $\ket{\psi}=\frac{\ket{0}+\ket{1}}{\sqrt{2}}$ yields the outcome 1 with probability $\braket{\psi|0}\braket{0|\psi}=\frac{1}{2}$. Similarly, the outcome -1 has probability $\frac{1}{2}$.
\end{example}

% Another feature that distinguishes quantum from classical is that that we can measure a qubit in basis aside from the standard basis. 
******************************************************


For our purposes the most important measurements are projective measurements with respect to an orthonormal basis, as in \textbf{Examples} \ref{measurement-standard basis} and \ref{theorem-distinguishing orthogonal states}.  In this paper, when we measure in this way, we will say we are "measuring with respect to an orthonormal basis."



As was suggested in the {\bf{Examples}} \ref{measurement-standard basis} and \ref{theorem-distinguishing orthogonal states} these measurements have some particularly nice properties.

 Specifically, if we measure the quantum state $\ket{\psi}$ with respect to the orthonormal basis $\{\ket{b_j}\}_{j=1}^d$, by making use of the completeness property the probability of observing outcome $j$ is given by,
 \begin{equation}
 p(j)=\braket{\psi|b_j}\braket{b_j|\psi}= \overline{\braket{b_j|\psi}}\braket{b_j|\psi}= |\braket{b_j|\psi}|^2.
 \end{equation}
 When this outcome is observed, the post-measurement state is $\ket{b_j}$, up to a global phase.  Thus, \textbf{Example} \ref{measurement-standard basis} is exactly this computation applied to the orthonormal basis $\{\ket{0},\ket{1}\}.$

**********************************************************


Another very nice property of projective measurements is that it is very easy to calculate the expected value of a given the observable M of a projective measurement:
\begin{eqnarray}
E(M)&=&\sum_m mp(m)\\
&=&\sum_m m \braket{\psi|P_m|\psi}\\
&=&\braket{\psi|(\sum_m mP_m)|\psi}\\
&=&\braket{\psi|M|\psi}
\end{eqnarray}

****************************************************
%----------------------------------------------------------------------------------------
%	SECTION 4
%----------------------------------------------------------------------------------------
\pagebreak
\section{Postulate 4: Composite Systems} \label{section: composite systems}
\begin{quote}
    \textbf{Postulate 4}:  The state space of a composite physical system is the tensor product of the component physical systems. Moreover, if we have $n$ systems numbered $1$ through $n$, where system $i$ is in the state $\ket{\psi_i}$, then the joint state of the total system is $\ket{\psi_1} \otimes \ket{\psi_1} \otimes \hdots \otimes \ket{\psi_n}$
\end{quote}

Postulate 4 makes clear the crucial role tensor products play in quantum mechanics.  This Postulate also enables us to define one of the most important ideas associated with composite quantum systems, that of \textit{entanglement}. 


\begin{definition}\label{definition: entanglement with state vector}
We say that a state of a composite system is \textit{separable} if it is the tensor product of two vectors in its component systems. Otherwise, we say the composite system is \textit{entangled}.
\end{definition}


\begin{example}
Consider the two qubit state 
\begin{equation}
    \ket{\psi}=\frac{\ket{00}+\ket{11}}{\sqrt{2}}.
\end{equation}

There are no single qubit states $\ket{a}, \ket{b}$ such that $\ket{\psi}=\ket{a}\otimes \ket{b}$, hence $\ket{\psi}$ is entangled.
\begin{proof}
Suppose for a contradiction, $\psi = \ket{a} \otimes \ket{b}.$  Then, 
$$\ket{a}=\alpha \ket{0}+\beta \ket{1}, \ket{b}=\gamma\ket{0}+\delta\ket{1}.$$
Then,
\begin{eqnarray}
\ket{a}\otimes \ket{b} &=& (\alpha \ket{0}+\beta \ket{1}) \otimes (\gamma\ket{0}+\delta\ket{1})\\
&=& \alpha \gamma \ket{00}+\alpha \delta \ket{01}+\beta \gamma\ket{10}+\beta\delta\ket{11}\\
&=&\frac{1}{\sqrt{2}}\ket{00}+0\ket{01}+0\ket{10}+\frac{1}{\sqrt{2}}\ket{11}\\
&=&\ket{\psi}.
\end{eqnarray}
Therefore $\alpha \gamma \neq 0$ and $\beta \delta \neq 0$, so all are nonzero.  But $\alpha \delta =0$ (and $\beta \gamma=0$), a contradiction.
\end{proof}
\end{example}
In {\bf{Chapter}} \ref{Chapter6-classification of entanglement} we classify entanglement completely.

\textcolor{red}{maybe add a few properties of entanglements just to give more intuition for the readers what will happen next}

