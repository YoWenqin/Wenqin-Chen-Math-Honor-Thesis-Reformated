% Chapter Template

\chapter{Density Matrices} % Main chapter title

\label{Chapter4-density matrix} % Change X to a consecutive number; for referencing this chapter elsewhere, use \ref{ChapterX}

In the last chapter we formulated the postulates of quantum mechanics using the language of state vectors. These could have been stated using \textit{density matrices} as well. We have already seen that a {\emph{fixed}} state vector for a system gives rise to a probability distribution when it is measured.  Density matrices allow one to describe a quantum system whose state is probabilistic {\emph{even prior to measurement}}.  Thus with density matrices probability enters the discussion at another level altogether.

This additional wrinkle is interesting, but is not used significantly in future chapters.  In the interest of completeness, we summarize some key ideas in the "density matrix" formulation.  The reader may choose to skip everything in the chapter after {\bf{Lemma}} \ref{lemma state into density}.

\begin{definition}
Let $\{\ket{\psi_1},\hdots,\ket{\psi_n}\}$ be states.  Suppose that a quantum system is in the state $\ket{\psi_i}$ with probability $p_i$, where  
\begin{equation}
\sum\limits_i^n p_i=1, p_i \geq 0.
\end{equation}
Then, we define the density matrix for the quantum system $\rho$, by 
\begin{equation}
    \rho:=\sum_i p_i \ket{\psi_i}\bra{\psi_i}.
\end{equation}
\end{definition}

Note that density matrices can be viewed as generalizing state vectors.  Let $\ket{\psi}$ be a state vector and consider the assignment
$$\ket{\psi} \xrightarrow{\iota} \ket{\psi}\bra{\psi}.$$

Clearly, the matrix $\ket{\psi}\bra{\psi}$ can be regarded as the density matrix for the system that is in the state $\ket{\psi}$ with probability $1$, i.e. the state vector $\ket{\psi}$.  The following lemma shows that this function is an inclusion for our purposes.

% \textcolor{blue}{ADD LINES TO PROOF BELOW} \textcolor{green}{done}


\begin{lemma}
\label{lemma state into density}
Let $\ket{\psi_1}, \ket{\psi_2}$ be state vectors in ${\mathbb{C}}^n$ with $\iota({\ket{\psi_1}})=\iota(\ket{\psi_2})$.\\
Then $\ket{\psi_1}=\ket{\psi_2}$ up to a global phase.
\end{lemma}
\begin{proof}
Let $\ket{\psi_1}=\icol{a_1\\ \vdots \\ a_n}, \ket{\psi_2}=\icol{b_1\\ \vdots \\ b_n}$. Then
\begin{gather}
\iota({\ket{\psi_1}})=\iota(\ket{\psi_2})\\
\Rightarrow \begin{pmatrix}
a_1 \Bar{a_1} && a_1 \Bar{a_2} && \hdots && a_1 \Bar{a_n}\\
&& && \vdots && \\
a_n \Bar{a_1} && a_n \Bar{a_2} && \hdots && a_n \Bar{a_n}\\              
\end{pmatrix}=\begin{pmatrix}
b_1 \Bar{b_1} && b_1 \Bar{b_2} && \hdots && b_1 \Bar{b_n}\\
&& && \vdots && \\
b_n \Bar{b_1} && b_n \Bar{b_2} && \hdots && b_n \Bar{b_n}\\ 
\end{pmatrix}\\
\Rightarrow 
\begin{cases}
a_1 \Bar{a_1}=b_1 \Bar{b_1}\\
a_2 \Bar{a_2}=b_2 \Bar{b_2}\\
\hdots\\
a_n \Bar{a_n}=b_n \Bar{b_n}\\
a_1 \Bar{a_2}=b_1 \Bar{b_2}\\
\hdots
\end{cases}
\Rightarrow \begin{cases}
a_1 = e^{i\theta_1}b_1\\
\hdots\\
a_n = e^{i\theta_n}b_n\\
a_1 \Bar{a_2}=b_1 \Bar{b_2}\\
\hdots
\end{cases}
\Rightarrow \begin{cases}
e^{i\theta_1}b_1 e^{-i\theta_2}\Bar{b_2}=b_1 \Bar{b_2}\\
\hdots\\
\end{cases}\\
\Rightarrow \theta_1 = \theta_2 = \hdots = \theta_n
\Rightarrow\ket{\psi_1}=e^{i\theta_1}\ket{\psi_2}
\end{gather}
\end{proof}



Since state vectors can only be determined up to a global phase, from a quantum mechanics perspective, the map $\iota$ is one-ton-one, hence state vectors are a special case of density matrices.  If we wish to emphasize that a state is given by a proper density matrix, we say the state is \textit{mixed}.  In similar fashion, states given by vectors are sometimes called {\emph{pure}}. 
% The material in this Chapter past this point will not be used in subsequent Chapters of the thesis, so the reader can freely skip ahead.

\textcolor{blue}{THE ABOVE IS WRITTEN FOR SUBSEQUENT SECTIONS.  IF THEY GET MOVED TO AN APPENDIX, THE SENTENCE ABOVE MUST BE REWRITTEN}

\begin{theorem}[Characterization of density matrix]
A matrix $\rho$ is the density matrix associated to some ensemble $\{p_i, \ket{\psi_i}\}$ if and only if it satisfies the conditions:
\begin{enumerate}
    \item (\textbf{Trace condition}) $tr(\rho)=1$
    \item (\textbf{Positive condition}) $\rho$ is a positive matrix \footnote{Positive operators is a special subclass of Hermitian operators. A positive operator A is definined to be an operator such that for any vector $\ket{v}$, $\langle \ket{v}, A\ket{v} \rangle \in \mathbb{R}_{\ge 0}$. It turns out that any positive operator has diagonal representation $\sum_i \lambda_i \ket{i}\bra{i}$ with non-negative eigenvalues $\lambda_i$.} 
\end{enumerate}
\end{theorem}

\begin{proof}
Given a density matrix $\rho=\sum_i p_i \ket{\psi_i}\bra{\psi_i}$. Then
\begin{equation}
    tr(\rho)=\sum_i p_i tr(\ket{\psi_i}\bra{\psi_i})=\sum_i p_i =1
\end{equation}
So the trace condition is satisfied. Notice here from section \ref{section:complex vector space} we know all eigenvalues of a projector like $\ket{\psi_i}\bra{\psi_i}$ are 0 or 1. Since trace is equal to the sum of eigenvalues, we have $tr(\ket{\psi_i}\bra{\psi_i})=1$. Now suppose $\ket{\varphi}$ is an arbitrary vector in state space. Then
\begin{equation}
    \braket{\varphi|\rho|\varphi}=\sum_i p_i \braket{\varphi|\psi_i} \braket{\psi_i|\varphi}=\sum_i p_i |\braket{\varphi|\psi_i}|^2 \ge 0
\end{equation}
So the positivity condition is satisfied.

Conversely, suppose $\rho$ is any operator satisfying the two conditions. Because of the positive condition, we have $\rho=\sum_j \lambda_j \ket{j}\bra{j}$, where the vectors $\ket{j}$ are orthogonal, and $\lambda_j$ are real, non-negative eigenvalues of $\rho$. Because of the trace condition, we have $\sum_j \lambda_j=1$. Let $p_i=\lambda_i$, so the probability of each $\ket{j}$ sums up to 1. So $\rho$ is essentially representing an ensemble of states $\{\lambda_j, \ket{j}\}$.
\end{proof}

\pagebreak

