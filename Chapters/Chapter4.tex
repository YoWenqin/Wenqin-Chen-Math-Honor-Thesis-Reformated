% Chapter Template

\chapter{Density Matrix as a More Useful Representation} % Main chapter title

\label{Chapter4-density matrix} % Change X to a consecutive number; for referencing this chapter elsewhere, use \ref{ChapterX}

We formulated quantum mechanics using the language of state vectors in the previous section. An alternative way is to use a tool called the \textit{density matrix}. It turns out to be equivalent to the state vector approach, but allows us to leverage tools in linear algebra more conveniently. It also allows for describing a quantum system whose state is not completely known, i.e. a quantum system which is in one of states $\{\ket{\psi_1},\hdots,\ket{\psi_n}\}$, with respective probabilities $\{p_1, p_2, \hdots, p_n\}$. We then call $\{p_i, \ket{\psi_i}\}$ an \textit{ensemble of pure states} and define the density matrix for the quantum system to be
\begin{equation}
    \rho:=\sum_i p_i \ket{\psi_i}\bra{\psi_i}
\end{equation}

\textcolor{red}{add an example of density matrix with probability distribution and its outcome probability}.

\begin{definition}[Density Matrix]
 Consider a quantum system with state space $\mathbb{C}^d$. A density matrix, commonly denoted as $\rho$, is a linear operator $\rho \in \mathbb{L}(\mathbb{C}^d, \mathbb{C}^d)$ such that:
 \begin{enumerate}
     \item $\rho \geq 0$, and
     \item $tr(\rho)=1$.
 \end{enumerate}
 
 If $rank(\rho)=1$, then $\rho$ is called a pure state, otherwise $\rho$ is mixed.
\end{definition}

\begin{definition}[Measuring a Density Matrix in a Basis]
Consider a quantum system in the state $\rho$. Measuring $\rho$ in the basis $\{\ket{b_j}\}_j$ results in outcome $j$ with probability $q_j=\braket{b_j | \rho |b_j}$.
\end{definition}

\begin{definition}[Projective Measurement] \label{projective measurement}
A projective measurement, also called a von Neumann measurement, is given by a set of orthogonal projectors $M_x=\Pi_x$ such that $\sum_x \Pi_x=\mathbb{I}$. For such a measurement, unless otherwise specified we will always use the default Kraus decomposition $A_x=\Pi_x$. The probability $q_x$ of observing measurement outcome x can then be expressed as\\
$q_x=tr(\Pi_x)$,
and the post-measurement states are \\
$\rho_{\ket{x}}=\frac{\Pi_x \rho \Pi_x}{tr(\Pi_x \rho)}$.
\end{definition}



% We can simply examine a classical bit to determine whether it is in the state 0 or 1. However, we cannot examine a qubit to determine its quantum state, that is, the value of $\alpha$ and $\beta$. Quantum mechanics tells us that we can only acquire much more restricted information about the quantum state. When we measure a qubit we get either the result 0, with probability $|\alpha|^2$, or the result 1, with probability $|\beta|^2$. Since the probabilities must sum to 1, $|\alpha|^2+|\beta|^2=1$. Thus, a qubit's state is a unit vector in a two-dimensional complex vector space. 

Measurement changes the state of a qubit, collapsing it from its superposition of $\ket{0}$ and $\ket{1}$ to the specific state consistent with the measurement result.  



\begin{definition}[Entanglement]
 Consider two quantum systems A and B. The joint state $\rho_{AB}$ is separable if there exists a probability distribution $\{p_i\}_i$, and sets of density matrices $\{\rho_i^A\}_i$, $\{\rho_i^B\}_i$ such that $\rho_{AB}=\sum_i p_i\rho_i^A\otimes\rho_i^B$.
 If there exists no such decomposition $\rho_{AB}$ is called entangled.
 If $\rho_{AB}=\ket{\psi}\bra{\psi}$ is a pure state, then $\psi_{AB}$ is separable if and only if there exists $\ket{\psi}_A, \ket{\psi}_B$ such that $\ket{\psi}_{AB}=\ket{\psi}_A \otimes \ket{\psi}_B$.
\end{definition}

\textcolor{red}{for most of what follows, we are only interested in pure states. maybe put the general definition in footnote.}

\textcolor{blue}{The No Cloning Theorem seems a bit out of place here.  Also, do we need this for something later?}
\begin{theorem}[No Cloning Theorem] \label{no-cloning thm}
Arbitrary qubits (or quantum states), unlike classical bits, cannot be copied. 
\end{theorem} 

\begin{proof} \cite{Wehner:notes}
For contradiction, assume there exists such unitary operation C that can copy any arbitrary qubits $\ket{\psi_1}, \ket{\psi_2}$. Then
\begin{equation*}
    C(\ket{\psi_1} \otimes \ket{0})=\ket{\psi_1} \otimes \ket{\psi_1}
\end{equation*}

\begin{equation*}
C(\ket{\psi_2} \otimes \ket{0})=\ket{\psi_2} \otimes \ket{\psi_2}  
\end{equation*}
    

Since C is unitary, we have $C^\dagger C=\mathbb{I}$, so
\begin{eqnarray*}
\braket{\psi_1|\psi_2}&=&\braket{\psi_1|\psi_2}\braket{0|0}\\
&=&(\ket{\psi_1} \otimes \ket{0})(\ket{\psi_2} \otimes \ket{0})\\
&=&(\ket{\psi_1} \otimes \ket{0})C^\dagger C(\ket{\psi_2} \otimes \ket{0})\\
&=&(\bra{\psi_1} \otimes \bra{\psi_1})(\ket{\psi_2} \otimes \ket{\psi_2})\\
&=&(\braket{\psi_1 |\psi_2})^2
\end{eqnarray*}

So $\braket{\psi_1 | \psi_2}=0$ or $1$. So $\ket{\psi_1}$ and $\ket{\psi_2}$ are either equal or othorgonal. So we can only clone states which are orthogonal to one another, but cloning an arbitrary quantum state is impossible. 
\end{proof}