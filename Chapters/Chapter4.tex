% Chapter Template

\chapter{Density Matrices} % Main chapter title

\label{Chapter4-density matrix} % Change X to a consecutive number; for referencing this chapter elsewhere, use \ref{ChapterX}


In the last chapter we formulated the postulates of quantum mechanics using the language of state vectors. These could have been stated using \textit{density matrices} as well. 

There are at least two motivations behind introducing density matrices. We have already seen that measurements yield different states with a probability distribution. The first motivation behind density matrices is to describe a quantum system whose state is probabilistic {\emph{even prior to measurement}}. You can imagine a source such as a machine that \emph{prepares} different quantum states subject to a probability distribution. Thus with density matrices, probability enters the discussion even before a system is measured.
In addition, since entangled joint state vectors such as $\ket{EPR}$ (see \eqref{eqn: epr}) cannot be written as a simple tensor of two ket vectors, it is unclear how to describe the state vectors of the component systems.  Density matrices provide the framework for doing this. 

\begin{definition}
Let $\{\ket{\psi_1},\hdots,\ket{\psi_n}\}$ be states, and let $p_i\geq 0$ with $\sum\limits_i p_i=1$.
Then, 
\begin{equation}
    \rho:=\sum_i p_i \ket{\psi_i}\bra{\psi_i}.
\end{equation}
is the density matrix corresponding to $\{(p_i, \ket{\psi_i})\}$. 
\end{definition}
The set called $\{(p_i, \ket{\psi_i})\}$ is called the \textit{ensemble} of states for the density matrix $\rho$.  When $\rho$ is a density matrix with ensemble of states $\{(p_i, \ket{\psi_i})\}$, we sometimes say the state $\ket{\psi_i}$ {\emph{is prepared with probability}} $p_i$.
% that a quantum system is in the state $\ket{\psi_i}$ with probability $p_i$, where  
% \begin{equation}
% \sum\limits_i^n p_i=1, p_i \geq 0.
% \end{equation}
% Then, we define the density matrix for the quantum system $\rho$, by 
% \begin{equation}
%     \rho:=\sum_i p_i \ket{\psi_i}\bra{\psi_i}.
% \end{equation}

\bigskip

Note that density matrices can be viewed as generalizing state vectors.  Let $\ket{\psi}$ be a state vector and consider the assignment $\iota$ given by
$$\ket{\psi} \xrightarrow{\iota} \ket{\psi}\bra{\psi}.$$

Clearly, the matrix $\iota(\ket{\psi})=\ket{\psi}\bra{\psi}$ can be regarded as the density matrix for the system that is in the state $\ket{\psi}$ with probability $1$, i.e. the state vector $\ket{\psi}$.  Notice that $\iota({\ket{\psi}})$ is a matrix of rank one with precisely nonzero eigenvalue ($\lambda=1$ corresponding to eigenvector $\ket{\psi}$).  We call the density matrix $\iota(\ket{\psi})$ a \textit{pure state}. Thus, a density matrix is a pure state if it is a one-dimensional projector. Otherwise, it is a \textit{mixed state}. We will consider pure states in this thesis.

The following lemma shows that the function $\iota$ is an inclusion for our purposes.

\begin{lemma}
\label{lemma state into density}
Let $\ket{\psi_1}, \ket{\psi_2}$ be state vectors in ${\mathbb{C}}^n$ with $\iota({\ket{\psi_1}})=\iota(\ket{\psi_2})$.\\
Then $\ket{\psi_1}=\ket{\psi_2}$ up to a global phase.
\end{lemma}
\begin{proof}
Let $\ket{\psi_1}=\icol{a_1\\ \vdots \\ a_n}, \ket{\psi_2}=\icol{b_1\\ \vdots \\ b_n}$. 
\bigskip

Then, $\iota({\ket{\psi_1}})=\iota(\ket{\psi_2})$ so
\begin{equation*}
\begin{pmatrix}
a_1 \Bar{a_1} && a_1 \Bar{a_2} && \hdots && a_1 \Bar{a_n}\\
&& && \vdots && \\
a_n \Bar{a_1} && a_n \Bar{a_2} && \hdots && a_n \Bar{a_n}\\              
\end{pmatrix}=\begin{pmatrix}
b_1 \Bar{b_1} && b_1 \Bar{b_2} && \hdots && b_1 \Bar{b_n}\\
&& && \vdots && \\
b_n \Bar{b_1} && b_n \Bar{b_2} && \hdots && b_n \Bar{b_n}\\ 
\end{pmatrix}
\end{equation*}

\noindent
Then for all $i$, $$a_i \Bar{a_i}=b_i \Bar{b_i},$$
and for all $i, j$ with $i \ne j$, 
\begin{equation}\label{eqn: different index}
    a_i \Bar{a_j}=b_i \Bar{b_j}
\end{equation}
Therefore, 
\begin{equation}\label{eqn: same index}
   a_i=e^{i\theta_i}b_i. 
\end{equation}
By substituting \eqref{eqn: same index} into \eqref{eqn: different index}, we get 
\begin{equation}
    e^{i\theta_i}b_i \overline{e^{j\theta_j}b_j}=b_i \Bar{b_j}
=e^{i\theta_i}b_i e^{-i\theta_j}\Bar{b_j}=e^{i(\theta_i-\theta_j)}b_i \Bar{b_j}.
\end{equation}
Thus, $\theta_i=\theta_j \pmod{2\pi}$ for all $i,j$, 
therefore $\ket{\psi_1}=\icol{e^{i\theta_1}b_1 \\ \vdots \\ e^{i\theta_1}b_n}=e^{i\theta_1}\ket{\psi_2}$ as required.
\end{proof}

Since state vectors can only be determined up to a global phase, from a quantum mechanics perspective, the map $\iota$ is one-ton-one, hence state vectors are a special case of density matrices. The postulates of quantum mechanics can be formulated in the language of density matrices. Here we just present the definition of measuring with respect to an orthonormal basis using density matrices. In other words, the following is the density matrix version of \textbf{Definition \ref{def: measurement in a basis state vector}}.


\begin{definition}
Consider a quantum system in a state in $\mathbb{C}^n$ given by density matrix $\rho$.  When we measure $\rho$ with respect to the basis $\{\ket{u_i}\}_{i=1}^n$, we observe outcome $i$ with probability 
\begin{equation} \label{eqn: measurement probability in density matrix}
    q_i=\braket{u_i | \rho |u_i}
\end{equation}
\end{definition}


A justification for \textbf{Definition \ref{eqn: measurement probability in density matrix}} as well as a description of how the four postulates can be reformulated for density matrices can be found in \textbf{Appendix \ref{AppendixB}}, but here is an example of measuring a density matrix with respect to an orthonormal basis.

\begin{example}
Consider a source that prepares the state $\ket{0}$ and $\ket{1}$ with probabilities $p_1=p_2=\frac{1}{2}$. Suppose we measure the corresponding mixed state
$$\rho=\frac{1}{2}(\ket{0}\bra{0}+\ket{1}\bra{1})=\frac{\mathbb{I}}{2},$$
with respect to the Hadamard basis $\{\ket{+}, \ket{-}\}$ (see \eqref{eqn: hadamard basis}). The probabilities of the outcomes are given by 
\begin{eqnarray}
    q_1&=&\braket{+|\rho|+}\\
    &=&\frac{1}{2}\braket{+|(\ket{0}\bra{0}+\ket{1}\bra{1})|+}\\
    &=&\frac{1}{2}\braket{+|0}\braket{0|+}+\frac{1}{2}\braket{+|1}\braket{1|+}\\
    &=&\frac{1}{2} \times \frac{1}{2} + \frac{1}{2} \times \frac{1}{2}\\
    &=&\frac{1}{2}
\end{eqnarray}
Similarly, $q_2=\braket{-|\rho|-}=\frac{1}{2}$
\end{example}


One important application is when a measurement on a {\emph{joint state}} is made in {\emph{one component only}}.  In other words, suppose that $\ket{\psi}$ in $\mathbb{C}^n \otimes \mathbb{C^n}$ is a composite state shared by Alice and Bob.  By using $\iota$, we may regard $\ket{\psi}$ as the density matrix $\rho=\ket{\psi}\bra{\psi}$.  Then, we can make sense of Alice makes a measurement with respect to an orthonormal basis $\{\ket{u_i}\}_{i=1}^n\subseteq {\mathbb{C}}^n$, by using of the family of operators $$\{\ket{u_i}\bra{u_i} \otimes \mathbb{I}_n\}_{i=1}^n.$$ 

Intuitively, this is reasonable because in our situation, only Alice makes a measurement, while Bob does nothing.  If Alice makes a measurement and observes outcome $i_0$ with end state $\ket{u_{i_0}}$, then the probability of each outcome for Bob should be the same as it was before Alice's measurement.  This can be checked by \eqref{eqn: general measurement probability density matrix}.

More applications of density matrices are discussed in \textbf{Appendix \ref{AppendixB}}.  In particular, the  \textit{partial trace} provides a tool to describe Alice's state when she shares a composite system with Bob.  Formal definitions of the partial trace as well as \textit{reduced density matrices} can be found in \textbf{Appendix \ref{AppendixB}}.




