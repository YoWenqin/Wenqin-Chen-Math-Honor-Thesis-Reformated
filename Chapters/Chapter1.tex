% Chapter 1

\chapter{Introduction} % Main chapter title

\label{Chapter1-introduction} % For referencing the chapter elsewhere, use \ref{Chapter1} 

%----------------------------------------------------------------------------------------

% Define some commands to keep the formatting separated from the content 
\newcommand{\keyword}[1]{\textbf{#1}}
\newcommand{\tabhead}[1]{\textbf{#1}}
\newcommand{\code}[1]{\texttt{#1}}
\newcommand{\file}[1]{\texttt{\bfseries#1}}
\newcommand{\option}[1]{\texttt{\itshape#1}}

It is a deep principle of nature that an observer necessarily interacts with the system he or she studies. Quantum mechanics is the branch of modern physics which investigates this and other similar phenomena. In this thesis we will study the application of quantum mechanics to cryptography and message transmission. Of particular interest will be the role played by {\emph{entanglement}}, a state where particles can be linked in such a way that interacting with one instantly affects the other. 

In cryptography, two parties, Alice and Bob, wish to send a message and protect it from the prying eyes of their adversary Eve. It is well known that if Alice and Bob make use of a one-time pad to encrypt their message, it is \emph{secure} in the sense that Eve gains no information about the message even if she intercepts the ciphered text. 
% For the main result of this thesis we will describe a mechanism exploiting entanglement which enables Alice and Bob to make a one-time pad at a distance, without the need to communicate during pad creation. 
% Let us briefly provide some background on quantum mechanics and cryptography.  
In the quantum world, rather surprisingly an unobserved particle may exist in all states simultaneously. Only when it is measured does its state {\emph{collapse}} to the observed (classical) state.  Two particles may be so {\emph{entangled}} that it is not possible to describe one physically independent of the other.  


In {\emph{quantum cryptography}}, one aims to exploit such curious phenomena in order to enhance security. For example, Alice and Bob may use the BB84 protocol to make a one-time pad in the case that they can communicate through both a classical channel (such as a phone line or the internet) and a quantum channel. This protocol uses quantum mechanics (though not entanglement) to detect a possible breach of security during pad creation.  

In this thesis, we will describe a cryptography protocol making use of entanglement \textcolor{green}{which allows Alice and Bob to construct a one-time pad, assuming share pairs of entangled particles that allow for \emph{maximal} correlation within each pair. For each pair of entangled particles distributed to Alice and Bob respectively, the goal is for Bob to infer Alice's measurement outcome (the state that Alice's particle collapses into after she measures her particle) by only measuring his own particle. In order to achieve such purpose, we first explore mathematical tools that allow us to quantify the strength of the correlation between two parties Alice and Bob when they share a composite quantum state. It turns out that only certain measurements allow Bob to make use of the correlation to infer Alice's measurement outcome with certainty. We therefore examine what are the conditions that allow for an effective measurement.}
%  without communicating, assuming that they have access to a {\emph{maximally}} entangled measuring device.  ***Raj: It isn't the measuring device that is maximally entangled. Rather, they have some device for making maximally entangled particles. My understanding is that this is possible, in the sense that you can do it in a high energy physics lab.*** This strategy for pad creation differs from BB84 in that Alice and Bob need not send information to each other during the creation of the pad.

This thesis will be organized as follows. 
\textcolor{green}{We will begin by introducing some fundamentals in quantum mechanics in the language of linear algebra in \textbf{Chapter \ref{Chapter2-preliminaries}, \ref{Chapter3-postulates}} and \textbf{\ref{Chapter4-density matrix}}. In \textbf{Chapter \ref{Chapter5-cryptography}}, we will explore how quantum cryptography might be provide better security than classical cryptography protocols currently used and provide a summary of two existing quantum cryptography protocols: BB84 and E91. In \textbf{Chapter \ref{Chapter6-classification of entanglement}}, we will examine mathematical tools that quantify the "strength" of correlation between two parties when they share an entangled quantum state. Finally, we will complete our story by making use of states that have the maximal correlation to create a one-time pad that enable secure information exchange between Alice and Bob and suggest the further directions worth exploring in \textbf{Chapter \ref{Chapter7-further discussion}}.}


% We will begin with preliminaries on quantum mechanics and cryptography, including an examination of existing quantum cryptography protocols such as BB84. Then, we will describe our mechanism for using shared states and an entangled density matrix to build a one-time pad remotely. Along the way, we will prove some theorems classifying the degree of quantum entanglement of a shared state. We will also make precise the way in which entangled density matrices allow for information exchange, and we will provide a physical interpretation for the degree of entanglement. 

% We will also continue to examine the case when the shared density matrix is not maximally entangled and will explore the correlation between the matrix and the system's ability to withstand errors. 
