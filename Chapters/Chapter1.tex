% Chapter 1

\chapter{Introduction} % Main chapter title

\label{Chapter1-introduction} % For referencing the chapter elsewhere, use \ref{Chapter1} 

%----------------------------------------------------------------------------------------

% Define some commands to keep the formatting separated from the content 
\newcommand{\keyword}[1]{\textbf{#1}}
\newcommand{\tabhead}[1]{\textbf{#1}}
\newcommand{\code}[1]{\texttt{#1}}
\newcommand{\file}[1]{\texttt{\bfseries#1}}
\newcommand{\option}[1]{\texttt{\itshape#1}}




It is a deep principle of nature that an observer necessarily interacts with the system he or she studies. Quantum mechanics is the branch of modern physics which investigates this and other similar phenomena. In this thesis we will study the application of quantum mechanics to cryptography and message transmission. Of particular interest will be the role played by {\emph{entanglement}}, the phenomenon that particles can be linked in such a way that interacting with one instantly effects the other. 

In cryptography, two parties, Alice and Bob, wish to send a message safely in the presence of a malicious third party Eve. It is well known that if Alice and Bob make use of a one-time pad to encrypt their message, it is \emph{secure} in the sense that Eve gains no information about the message even if she intercepts the ciphered text. In the quantum world, rather surprisingly an unobserved particle may exist in all states simultaneously. Only when it is measured does its state {\emph{collapse}} to the observed (classical) state.  Two particles may be so {\emph{entangled}} that it is not possible to describe one physically independent of the other.  In {\emph{quantum cryptography}}, one aims to exploit quantum mechanics in order to enhance security. For example, Alice and Bob may use the BB84 protocol (see \textbf{Chapter \ref{Chapter5-cryptography}}) to make a one-time pad in the case that they can communicate through both a classical channel, such as a phone line, as well as a \textit{quantum channel}. This protocol uses quantum mechanics to detect a possible breach of security during pad creation.  



% describe a cryptography protocol making use of entanglement which allows Alice and Bob to construct a one-time pad, assuming they share pairs of entangled particles that allow for \emph{maximal} correlation within each pair.
In this thesis, we define a mathematical tool which we use to classify the strength of entanglement between two states. We focus on the case where Alice and Bob share a entangled state, which only she measures.  In this situation, we describe precisely the way in which our tool measures the strength of the correlation between Alice's and Bob's parts of their shared system after her measurement. We then characterize exactly when Bob can {\emph{use}} this correlation to know what Alice has measured, even though they may be far away.  Lastly, we use this idea for the creation of a remote one-time pad.

%  Then we analyze when we can make use of the correlation and apply that to one time pad.
% For each pair of entangled particles distributed to Alice and Bob respectively, the goal is for Bob to infer Alice's measurement outcome (the state that Alice's particle collapses into after she measures her particle) by only measuring his own particle. In order to achieve such purpose, we create a mathematical tool that allow us to quantify the strength of the correlation between Alice and Bob when they share a composite quantum state. It turns out that only certain measurements allow Bob to make use of the correlation to infer Alice's measurement outcome with certainty. We therefore examine what are the conditions that allow for an effective measurement.

%  without communicating, assuming that they have access to a {\emph{maximally}} entangled measuring device.  ***Raj: It isn't the measuring device that is maximally entangled. Rather, they have some device for making maximally entangled particles. My understanding is that this is possible, in the sense that you can do it in a high energy physics lab.*** This strategy for pad creation differs from BB84 in that Alice and Bob need not send information to each other during the creation of the pad.

This thesis will be organized as follows.  We begin by introducing some preliminaries including the necessary linear algebra and the postulates of quantum mechanics in \textbf{Chapters \ref{Chapter2-preliminaries}} and  \textbf{\ref{Chapter3-postulates}}. In \textbf{Chapter \ref{Chapter4-density matrix}}, we discuss density matrices, a mathematical generalization of state vectors in quantum systems. In \textbf{Chapter \ref{Chapter5-cryptography}}, we introduce cryptography, and briefly explore why quantum cryptography improves classical cryptography protocols currently in use.  In this chapter we also include a quick summary of two existing quantum cryptography protocols, BB84 and E91. In \textbf{Chapter \ref{Chapter6-classification of entanglement}}, we present our main results. We define a tool that quantifies the strength of correlation between Alice's and Bob's states when they are entangled. We then describe how to make use of this correlation to create a one-time pad.  We finish by suggesting future research directions in \textbf{Chapter \ref{Chapter7-further discussion}}.


% We will begin with preliminaries on quantum mechanics and cryptography, including an examination of existing quantum cryptography protocols such as BB84. Then, we will describe our mechanism for using shared states and an entangled density matrix to build a one-time pad remotely. Along the way, we will prove some theorems classifying the degree of quantum entanglement of a shared state. We will also make precise the way in which entangled density matrices allow for information exchange, and we will provide a physical interpretation for the degree of entanglement. 

% We will also continue to examine the case when the shared density matrix is not maximally entangled and will explore the correlation between the matrix and the system's ability to withstand errors. 
