% Chapter Template

\chapter{Further Discussion} % Main chapter title

\label{Chapter7-further discussion} % Change X to a consecutive number; for referencing this chapter elsewhere, use \ref{ChapterX}

This thesis offers a possible extension to the BB84 and E91 protocol, giving rise to multiple interesting directions worth exploring; here I will just present two of them.

The first of the two directions is to study the relation between the coefficient matrix and the likelihood of an eavesdropper, Eve, successfully hacking the system. What if some errors or noises are introduced into the quantum communication channel and how might that affect the knowledge Eve gains about the one-time pad? To better answer these questions, not only is a thorough study on Bell's inequality \cite{bell1964} needed, but also tools to quantify uncertainty in message transmission, such as entropy \cite{renner2008security}. From there, it is worth investigating whether \textit{privacy amplification}, a method to further limit Eve's potential knowledge gain can be applied to our protocol in \textbf{Section \ref{section: application to a one-time pad}} \cite{bennett1995generalized}.

The second further direction is to expand our classification of entanglement to mixed states. To go from pure states, we will need to use density matrices extensively, and learn about other types of measurements other than projective measurements in \textbf{Appendix \ref{AppendixA}}. In particular, the Positive Operator Valued Measure, or known as POVM, is a mathematically convenient tool to study general measurements, when we only care about the measurement statistics, but not about the post-measurement states (see \cite{Nielsen}). For a generalized measurement described by a set of measurement operators $\{M_m\}$ we define a set of operators $\{{E_m}\}$ where
$E_m=M_m^\dagger M$. We can check that the completeness equation required for \textbf{Postulate 3} in \textbf{Section \ref{section: postulate 3}} is satisfied. The probability of getting outcome $m$ therefore becomes 
$$p(m)=\langle \psi | E_m | \psi \rangle$$
We call the operators $\{{E_m}\}$ the \textit{POVM elements} associated with the measurement.
Here is an example\footnote{This example is adapted from page 92 in \cite{Nielsen}.} of a general measurement or POVM that's not projective measurement or measurement with a basis. Suppose Alice gives Bob a qubit in one of two states $\ket{\psi_1}=\ket{0}, \ket{\psi_2}=\ket{+}$. These two states are not orthogonal, so projective measurements won't be able to distinguish them reliably. A POVM containing the following three elements proves useful ($c_i$ is a scalar),
\begin{gather}
    E_1=c_1\ket{1}\bra{1}\\
    E_2=c_2\ket{+}\ket{-}\\
    E_3=\mathbb{I}-E_1-E_2
\end{gather}
Suppose Bob is given $\ket{\psi_1}$, then we have $\braket{\psi_1|E_1|\psi_1}=0$, so if his POVM measurement result is $E_1$ then Bob can conclude that he received $\ket{\psi_2}$. Similarly, if his outcome is $E_2$, he can conclude that he received $\ket{\psi_1}$. However, if his measurement outcome is $E_3$, he can infer nothing. In addition to offering insights into measurements with respect to non-orthonormal bases, POVM also provides a way to describe how a projective measurement on a larger system affects its subsystems. For a more thorough study on POVM, see \cite{brandt1999positive}.


% \textcolor{blue}{I added some possible things in no particular order}

% \begin{itemize}
% \item Replacing pure states by mixed states in the classification of entanglement computation.
% \item Explicit computation on "general case" of maximally entangled states whose coordinate matrices are not scalar multiples of unitary matrices with a "randomly" selected basis.
% \item Probability that a randomly selected maximally entangled state has a coordinate matrix that is a scalar multiple of a unitary matrix.
% \item Anything about the entanglement "decaying".
% \end{itemize}

