% Chapter Template

\chapter{Classical and Quantum Cryptography} % Main chapter title

\label{Chapter5-cryptography} % Change X to a consecutive number; for referencing this chapter elsewhere, use \ref{ChapterX}


\begin{quote}
\textit{"The enemy knows the system."} 
\bigskip
\hfill \textit{-- Claude Shannon}
\end{quote}

Broadly speaking, cryptography is the practice of communication secretly in the presence of third parties who wish to intercept or corrupt the communications. The two communicating parties, Alice and Bob wish to exchange information, while Eve attempts to read their messages.

Encryption schemes can be classified grossly as either \emph{private key cryptography} or \emph{public key cryptography}. The distinction being that in private key cryptography, the same secret key is used for Alice's encryption and Bob's decryption, whereas in public key cryptography, one key is used to encrypt and is made available to the public, while a different secret key is used to decrypt.

Currently, pubic key cryptosystems such as the \emph{RSA cryptosystem} are the most widely used cryptographic protocol.  Public key cryptosystems don't rely on Alice and Bob sharing a secret key {\emph{in advance}}. Instead, Bob generates a pair of keys, and makes one of them available to the public. Alice can use that public key to encrypt her message, which Bob can decipher with the use of his other key, which remains private. Security in this situation, depends on it being extremely difficult for Eve to crack Bob's private key, even with access to the public key.  In practice, this typcially relies on a classical computer's difficulty in solving the problem of factoring when large primes are involved. It's worth noting that {\emph{Shor's fast algorithm}} on a quantum computer can easily break RSA.

\begin{figure}[h]
    \centering
    \includegraphics[scale=0.6]{"schematic of a general crypto system".png}
    \caption{Schematic of a general secrecy system from \cite{shannon1949communication}}
    \label{fig:general crypto schematic}
\end{figure}

In private key cryptosystems, a private key is used by Alice to encrypt the secret message she wants to send to Bob, who will also need the private key to decrypt the message. Of course, in this situation, the very act of distributing a key from Alice to Bob is dangerous because Eve might be eavesdropping at the moment the key is distributed. In this chapter, we will introduce some of the main concepts in cryptography.  Then we will illustrate some ways in which Alice and Bob can exploit the laws of quantum mechanics to establish better security. 

% \textcolor{blue}{How about one-time pad here first.  Also this is a good place for a picture of Alice and Bob.  We should be able to find a schematic with Alice and Bob somewhere} developed in \cite{shannon1949communication}.

We begin with some preliminary concepts.  Formally, an {\emph{encryption}} is given by an encryption function 
$$Enc(k,m)=e ,$$
which takes a key $k$ and the message $m$ and maps it to some encrypted message $e$. The original message $m$ is also called the {\emph{plaintext}}, whereas $e$ is refered to as the {\emph{ciphertext}}. Similarly, a {\emph{decryption function}} $$Dec(k,e)=m$$
that takes the key $k$ and the ciphertext $e$ as arguments, and returns the plaintext. 

If $m$ is a message, then p(m) is the probability of guessing one message correctly. For example, say the message an arbitrary 10-bit message picked uniformly randomly, then $p(m)=\frac{1}{2^10}$.
\begin{definition} \label{def: shannon}
An encryption scheme (Enc, Dec) is \textit{secret}, or \textit{secure} if and only if for all prior distributions p(m) over messages, and all messages m, we have
\begin{equation}
    p(m)=p(m|e),
\end{equation}
where $e=Enc(k,m)$, and $p(m|e)$ is the conditional probability of m given e.

An encryption scheme (Enc, Dec) is correct if and only if for all possible messages m, and all possible keys k, we have m=Dec(k, Enc(k,m)).
\end{definition}
In cryptography the goal is to design protocols that are both secure and correct, which means that not only must the ciphertext be decipherable, but also that if Eve sees the ciphertext, she knows nothing more than anyone else who hasn't seen the ciphertext at all. The following theorem shows that this is only possible when the number of possible keys is large relative to the number of possible messages.
\begin{theorem}
If the encryption scheme (Enc, Dec) is secure and correct, then the number of possible keys |K| is at least as large as the number of possible messages |M|, that is, $|K| \ge |M|$.
\end{theorem}
For the proof, see p.18 of \cite{Wehner:notes}. 


\textcolor{blue}{Insert a couple of sentences on one-time pads here.  Basically say that if Alice and Bob use a one-time pad that is created securely, their message is unbreakable}

\begin{definition}
The classical one-time pad is an encryption scheme in which the encryption of a message $m \in \{0, 1\}^n$ using the key $k \in \{0, 1\}^n$ is given by
\begin{equation}
    Enc(k,m)=(m \oplus k)=(m_1 \oplus k_1, m_2 \oplus k_2, \hdots, m_n \oplus k_n)=(e_1, \hdots, e_n)=e
\end{equation}
where $m_i \oplus k_i = m_i + k_i mod 2$. The decryption is given by
\begin{equation}
Dec(k, e)=e \oplus k = (e_1 \oplus k_1, e_2 \oplus k_2, \hdots, e_n \oplus k_n)
\end{equation}
\end{definition}
Dec(k,e) is apparently the inverse of Enc(k,m) since $m_i \oplus k_i \oplus k_i=m_i$, so the one-time pad scheme is correct. It also satisfies Shannon's \textbf{Definition \ref{def: shannon}} for security. The proof can be found in section 1.8.1 of \cite{Wehner:notes}.

\begin{example}
Say Alice wants to transmit a message "hello" to Bob and uses the conventional 8-bit ASCII encoding \footnote{For ASCII encoding, refer to \url{https://en.wikipedia.org/wiki/ASCII}} for each character, the encoding process can be illustrated by \textbf{Table  \ref{table: one-time pad encoding}}. The actual ciphertext that Alice is going to send to Bob is 0000 1011 0001 0111 0001 0101 0001 1100 0001 1011.

\begin{table}[h]
\centering 
\begin{tabular}{|c|c|c|c|c|c|}
\hline
character & h & e & l & l & o \\ \hline
plaintext in bits & 0110 1000 & 0110 0101 & 0110 1100 & 0110 1100 & 0110 1111 \\ \hline
one-time pad & c & r & y & p & t \\ \hline
one-time pad in bits & 0110 0011 & 0111 0010 & 0111 1001 & 0111 0000 & 0111 0100 \\ \hline
ciphertext & 0000 1011 & 0001 0111 & 0001 0101 & 0001 1100 & 0001 1011 \\ \hline
\end{tabular}
\caption{Alice's encoding process from "hello" to the ciphertext represented by a string of bits}
\label{table: one-time pad encoding}
\end{table}

Bob's decipher process using the same one-time pad can be illustrated by \textbf{Table \ref{table: one-time pad decoding}}:
\begin{table}[h] 

\begin{tabular}{|c|c|c|c|c|c|}
\centering
\hline
ciphertext & 0000 1011 & 0001 0111 & 0001 0101 & 0001 1100 & 0001 1011 \\ \hline
one-time pad in bits & 0110 0011 & 0111 0010 & 0111 1001 & 0111 0000 & 0111 0100 \\ \hline
plaintext in bits & 0110 1000 & 0110 0101 & 0110 1100 & 0110 1100 & 0110 1111 \\ \hline
plaintext in characters & h & e & l & l & o \\ \hline
\end{tabular}
\caption{Bob's decipher process from the ciphertext back to "hello"}
\label{table: one-time pad decoding}
\end{table}
\end{example}




At the end of \textbf{Chapter \ref{Chapter6-classification of entanglement}}, we will create a classical one-time pad making use of quantum mechanics.

\begin{definition} \label{def: quantum one-time pad}
The quantum one-time pad is an encryption scheme for qubits. To encrypt, Alice applies $X^{k_1} Z^{k_2}$ to qubit $\rho$ and sends the resulting state to Bob. To decrypt, Bob applies the inverse $(X^{k_1} Z^{k_2})^\dagger$ to obtain $\rho$, where X and Z are the Pauli-X and Pauli-Z matrix respectively. i.e.
\begin{equation}
    X=\begin{pmatrix}
    0 && 1\\
    1 && 0
    \end{pmatrix}, 
    Z=\begin{pmatrix}
    1 && 0\\
    0 && -1
    \end{pmatrix}
\end{equation}
\end{definition}

\textcolor{blue}{Include example of this computation.  How does this compare to the classical one-time pad?  Also, does the No Cloning Theorem get used for something later?}

\begin{theorem}[No Cloning Theorem] \label{no-cloning thm}
Arbitrary qubits (or quantum states), unlike classical bits, cannot be copied. 
\end{theorem} 

\begin{proof} \cite{Wehner:notes}
For contradiction, assume there exists such unitary operation C that can copy any arbitrary qubits $\ket{\psi_1}, \ket{\psi_2}$. Then
\begin{equation*}
    C(\ket{\psi_1} \otimes \ket{0})=\ket{\psi_1} \otimes \ket{\psi_1}
\end{equation*}

\begin{equation*}
C(\ket{\psi_2} \otimes \ket{0})=\ket{\psi_2} \otimes \ket{\psi_2}  
\end{equation*}
    

Since C is unitary, we have $C^\dagger C=\mathbb{I}$, so
\begin{eqnarray*}
\braket{\psi_1|\psi_2}&=&\braket{\psi_1|\psi_2}\braket{0|0}\\
&=&(\ket{\psi_1} \otimes \ket{0})(\ket{\psi_2} \otimes \ket{0})\\
&=&(\ket{\psi_1} \otimes \ket{0})C^\dagger C(\ket{\psi_2} \otimes \ket{0})\\
&=&(\bra{\psi_1} \otimes \bra{\psi_1})(\ket{\psi_2} \otimes \ket{\psi_2})\\
&=&(\braket{\psi_1 |\psi_2})^2
\end{eqnarray*}

So $\braket{\psi_1 | \psi_2}=0$ or $1$. So $\ket{\psi_1}$ and $\ket{\psi_2}$ are either equal or othorgonal. So we can only clone states which are orthogonal to one another, but cloning an arbitrary quantum state is impossible. 
\end{proof}

%----------------------------------------------------------------------------------------
%	SECTION 1
%----------------------------------------------------------------------------------------

\pagebreak

\section{Bell-Nonlocality and CHSH game} \label{section: bell-nonlocality}

\textcolor{blue}{What is locality?}
\textcolor{green}{The assumption that Alice performing her measurement does not influence the result of Bob's measurement - this assumption is incorrect in the quantum world, as validated by Bell's experiment}
\textcolor{blue}{I changed this part around.  Definitely something like the current first paragraph should come first, but I'm not sure where to include what the rules are for such a game.  The next two paragramphs should be synthesized a bit }


In many collaborative games, two players using a {\emph{quantum strategy}} can achieve a higher winning probability than if they use a {\emph{classical strategy}}. Because of this, many probabilistic games can be thought of as tests for entanglement, and therefore play a key role in quantum key distribution. In this context, when using a \textit{quantum strategy}, Alice and Bob can pick a state $\rho_{AB}$ to share, and agree on measurements to perform depending on their respective questions. In a classical game ...

Bell proved that the predictions of quantum theory are incompatible with those of any classical theory satisfying a natural notion of locality \cite{bell1964}. The bell-nonlocality can be understood by a non-local game between Alice and Bob, in which they are asked questions $x,y \in \{0,1\}$ respectively. They respond with asnwers $a, b \in \{0,1\}$ respectively. They are allowed to agree on a strategy beforehand, but are not allowed to communicate during the game. Alice and Bob win the game if and only if 
\begin{equation*}
 a+b (mod 2)=xy    
\end{equation*}

The winning probability is given by 
\begin{equation*}
    p^{CHSH}_{win} = \frac{1}{4}\sum_{x,y \in \{0,1\}} \sum_{a, b \: such \: that \: a+b \: mod 2 = xy} p(a,b|x,y)
\end{equation*}
where $p(a,b|x,y)$ is the probability that Alice and Bob answer a and b given questions x and y.



\textcolor{blue}{The equation above looks funny}

\begin{figure}[h]
    \centering
    \includegraphics[scale=0.8]{"non-local game".png}
    \caption{Alice and Bob in a non-local CHSH guessing game}
    \label{fig:non-local game}
\end{figure}



%-----------------------------------
%	SUBSECTION 1
%-----------------------------------
\subsection{Classical Strategy}

A classical strategy is simply given by functions $f_A(x)=a$ and $f_B(y)=b$. By checking all possible classical strategies in the following table, we see the maximum winning probability is $\frac{3}{4}$.

\begin{table}[h]
\centering
\begin{tabular}{|c|c|c|c|c|}
\hline
 $p(a,b|x,y)$ & \begin{tabular}[c]{@{}l@{}}a=0 if x=0\\ a=1 if x=1\end{tabular} & \begin{tabular}[c]{@{}l@{}}a=1 if x=0\\ a=0 if x=1\end{tabular} & a=1 & a=0 \\ \hline
\begin{tabular}[c]{@{}l@{}}b=0 if y=0\\ b=1 if y=1\end{tabular} & $\frac{1}{4}$ & $\frac{3}{4}$ & $\frac{1}{4}$ & $\frac{3}{4}$ \\ \hline
\begin{tabular}[c]{@{}l@{}}b=1 if y=0\\ b=0 if y=1\end{tabular} & $\frac{3}{4}$ & $\frac{1}{4}$ & $\frac{3}{4}$ & $\frac{1}{4}$ \\ \hline
b=1 & $\frac{1}{4}$ & $\frac{3}{4}$ & $\frac{3}{4}$ & $\frac{1}{4}$ \\ \hline
b=0 & $\frac{3}{4}$ & $\frac{1}{4}$ & $\frac{1}{4}$ & $\frac{3}{4}$ \\ \hline
\end{tabular}
\end{table}

%-----------------------------------
%	SUBSECTION 2
%-----------------------------------

\subsection{Quantum Strategy}
Suppose Alice and Bob share an EPR pair, where we label the qubit held by Alice A and the one held by Bob B: 
\begin{equation*}
    \ket{\psi_{AB}}=\frac{1}{\sqrt{2}}(\ket{0}_A \ket{0}_B+\ket{1}_A \ket{1}_B)
\end{equation*}

Suppose now that when x = 0, Alice measures her qubit in the standard basis $\{\ket{0}, \ket{1}\}$. When x = 1, Alice measures her qubit in the Hadarmard basis $\{\ket{+}, \ket{-}\}$, where
\begin{gather*}
    \ket{+}=\frac{1}{\sqrt{2}}(\ket{0}+\ket{1})\\
    \ket{-}=\frac{1}{\sqrt{2}}(\ket{0}-\ket{1})
\end{gather*}

Suppose when y=0, Bob measures his qubit in the basis $\{\ket{v_1}, \ket{v_2}\}$, where
\begin{gather*}
    \ket{v_1}=cos(\frac{\pi}{8})\ket{0}+sin(\frac{\pi}{8})\ket{1}\\
    \ket{v_2}=-sin(\frac{\pi}{8})\ket{0}+cos(\frac{\pi}{8})\ket{1}  
\end{gather*}
and when y = 1, Bob measures her qubit in $\{\ket{w_1}, \ket{w_2}\}$, where
\begin{gather*}
    \ket{w_1}=cos(\frac{\pi}{8})\ket{0}-sin(\frac{\pi}{8})\ket{1}\\
    \ket{w_2}=sin(\frac{\pi}{8})\ket{0}+cos(\frac{\pi}{8})\ket{1}   
\end{gather*}
Then consider the probability of winning, conditioned on the four different possible combinations of x, y:
\begin{eqnarray*}
p_{win|x=0,y=0}&=&p(a=0, b=0|x=0, y=0)+p(a=1, b=1|x=0, y=0)\\
&=&\braket{0 v_1|\psi}\braket{\psi|0v_1}+\braket{1 v_2|\psi}\braket{\psi|1v_2}\\
&=&|\braket{0 v_1|\psi}|^2 + |\braket{1 v_2|\psi}|^2\\
&=&|\frac{1}{\sqrt{2}}(\braket{0v_1|00}+\braket{0v_1|11})|^2+|\frac{1}{\sqrt{2}}(\braket{1v_2|00}+\braket{1v_2|11})|^2\\
&=&\frac{1}{2}|cos(\frac{\pi}{8})\braket{00|00}+sin(\frac{\pi}{8})\braket{01|00}+cos(\frac{\pi}{8})\braket{00|11}+cos(\frac{\pi}{8})\braket{01|11}|^2\\
&+&\frac{1}{2}|-sin(\frac{\pi}{8})\braket{10|00}+cos(\frac{\pi}{8})\braket{11|00}-sin(\frac{\pi}{8})\braket{10|11}+cos(\frac{\pi}{8})\braket{11|11}|^2\\
&=&cos^2(\frac{\pi}{8})
\end{eqnarray*}

Similarly,
\begin{eqnarray*}
p_{win|x=0,y=1}&=&p(a=0, b=0|x=0, y=1)+p(a=1, b=1|x=0, y=1)\\
&=&|\braket{0 w_1|\psi}|^2 + |\braket{1 w_2|\psi}|^2\\
&=&(\frac{1}{\sqrt{2}}cos\frac{\pi}{8})^2+(\frac{1}{\sqrt{2}}cos\frac{\pi}{8})^2\\
&=&cos^2 \frac{\pi}{8}
\end{eqnarray*}
\begin{eqnarray*}
p_{win|x=1,y=0}&=&p(a=0, b=0|x=1, y=0)+p(a=1, b=1|x=1, y=0)\\
&=&|\braket{+ v_1|\psi}|^2 + |\braket{- v_2|\psi}|^2\\
&=&\frac{1}{2}(\frac{1}{\sqrt{2}}cos\frac{\pi}{8}+\frac{1}{\sqrt{2}}sin\frac{\pi}{8})^2+\frac{1}{2}(\frac{1}{\sqrt{2}}cos\frac{\pi}{8}+\frac{1}{\sqrt{2}}sin\frac{\pi}{8})^2\\
&=&\frac{1}{2}(cos\frac{\pi}{8}+sin\frac{\pi}{8})^2\\
&=&cos^2 \frac{\pi}{8}
\end{eqnarray*}
\begin{eqnarray*}
p_{win|x=1,y=1}&=&p(a=1, b=0|x=1, y=1)+p(a=0, b=1|x=1, y=1)\\
&=&|\braket{- w_1|\psi}|^2 + |\braket{+ w_2|\psi}|^2\\
&=&\frac{1}{2}(cos\frac{\pi}{8}+sin\frac{\pi}{8})^2\\
&=&cos^2 \frac{\pi}{8}
\end{eqnarray*}
So the probability of winning is the average of the four above:
\begin{equation*}
    p_{win}=\frac{1}{4}\sum_{x,y}p_{win|x,y}=cos^2 \frac{\pi}{8} \approx 0.85
\end{equation*}

%----------------------------------------------------------------------------------------
%	SECTION 2
%----------------------------------------------------------------------------------------

The increased probability of winning by making use of a quantum strategy suggests that quantum mechanics could be exploited by either side of the cryptography arms race.  In the next section, we discuss a famous example of this.




\pagebreak

\section{BB84 Protocol}

\begin{figure}[h]
    \centering
    \includegraphics[scale=0.6]{"bb84 encoding".png}
    \caption{BB84 Bit Encoding \protect\footnotemark}
    \label{fig:BB84 bit encoding}
\end{figure}
\footnotetext{Figure from \url{https://www.cse.wustl.edu/~jain/cse571-07/ftp/quantum/\#Gisin02}}

In 1984 Charles Bennett and Gilles Brassard published the first Quantum Key Distribution protocol \cite{Bennett1984} now referred to as the BB84 Protocol. For Alice to transmit a message to Bob, she not only chooses the bit to send, but also a basis out of the standard basis and the Hadamard basis. In other words, if she chooses the standard basis (rectilinear basis), she will encode the classical bit 0 with $\ket{0}$ ($0^\circ$ in rectilinear basis), and the bit 1 with $\ket{1}$ ($90^\circ$ in rectilinear basis). If she chooses the Hadamard basis (diagonal basis), she will encode the bit 0 with $\ket{+}$ ($45^\circ$ in the diagonal basis), and the bit 1 with $\ket{-}$ ($135^\circ$ in the diagonal basis). 

In the first phase, Alice attempts to send a random string of bits to Bob by encoding each bit with a photon of the corresponding polorization as described in \ref{fig:BB84 bit encoding} over a quantum channel. For each photon received, Bob will measure the photon with a randomly chosen basis between the rectilinear basis and the diagonal basis. As explained in Chapter \ref{Chapter3-postulates}, only if Bob chose the same basis as Alice would he be able to get for certainty the same measurement result as what Alice sent. Otherwise, the result he get would be random.

In the second phase, Bob notifies Alice over any channel (can totally be an insecure classical channel) which basis he picked for each incoming photon. In response, Alice reports back to Bob whether he chose the correct basis for each photon. Then, Alice and Bob both discard bits corresponding to photons measured with different bases. Assuming no errors occurred and no one manipulated the photons, Bob and Alice now have the same string of bits, and that is the private key distributed from Alice to Bob. The process is illustrated in \ref{fig:BB84 bit example}.

\begin{figure}[h]
    \centering
    \includegraphics[scale=0.35]{"bb84 example".png}
    \caption{An example of distributing a key of bits from Alice to Bob. The result of secret key established is 0101.\protect\footnotemark}
    \label{fig:BB84 bit example}
\end{figure}
\footnotetext{Figure from \url{https://en.wikipedia.org/wiki/Quantum_key_distribution\#BB84_protocol:_Charles_H._Bennett_and_Gilles_Brassard_(1984)}}

To verify whether an eavesdropper Eve has tapped into the communication, Alice and Bob picked a random subset of their shared secret key and compare if they actually agree. If yes, they can discard those bits and have the remaining bits be the shared secret key. Otherwise, it means Eve attempts to infer the key. Because of the no-cloning theorem, she has to measure the photons sent by Alice before passing them on to Bob. She cannot replicate the photon, pass the original photon to Bob and keep the replicated photon to herself. Since Eve does not know which basis Alice picked to encode the bit, Eve has to guess. If she guesses wrongly and measures on the wrong basis, the information encoded on the other basis is lost. So when the photon reaches Bob, his measurement result will be random and he infer the wrong bit. Say Eve chooses the measurement basis incorrectly on average $\frac{1}{2}$ of the time, then Bob (using the same basis as Alice) will get the bit wrong $\frac{1}{2} \times \frac{1}{2}=\frac{1}{4}$ of the time). If m bits are compared, the probability of Eve being undetected is $(\frac{3}{4})^m$, which approaches 0 as m gets bigger.

%----------------------------------------------------------------------------------------
%	SECTION 4
%----------------------------------------------------------------------------------------
\pagebreak


\section{E91 Protocol} \label{section: e91}
\begin{figure}[h]
    \centering
    \includegraphics[scale=0.6]{"e91 illustrated".png}
    \caption{An illustration of a Quantum Crypto Protocol Utilizing Entanglement: E91 Protocol \protect\footnotemark}
    \label{fig:E91 Illustrated}
\end{figure}
\footnotetext{Figure from \url{https://www.cse.wustl.edu/~jain/cse571-07/ftp/quantum/\#Gisin02}}
In 1991, Eckert describes a quantum channel where there is a single source that emits pairs of entangled particles \cite{Ekert1991}. As shown in figure \ref{fig:E91 Illustrated}, Alice and Bob then each receive one particle from every pair emitted from the source. Then they each choose a random basis out of three bases to measure on their received particles. 

\begin{figure}[h]
    \centering
    \includegraphics[scale=0.4]{"e91 bases annotated".png}
    \caption{E91 Encoding}
    \label{fig:e91 encoding}
\end{figure}

The three bases to choose from are $Z_0, Z_{\frac{\pi}{4}}, Z_{\frac{\pi}{2}}$, where $Z_\theta$ is the standard basis rotated counterclockwise by $\theta$ (See \textbf{Figure \ref{fig:e91 encoding}}). In other words, if Alice or Bob choose $Z_0$, she will encode the classical bit 0 with  $0^\circ$ in $Z_0$ and encode the bit 1 with $90^\circ$ in $Z_0$. If they choose $Z_{\frac{\pi}{4}}$, classical bits 0, 1 will be encoded with $45^\circ$ and $135^\circ$ respectively. If they choose $Z_{\frac{\pi}{2}}$, 0 and 1 will be encoded with $90^\circ$ and $180^\circ$ respectively.

Alice and Bob each keeps the choice of basis private until the measurements are completed. As in BB84, they would then discuss in a classical channel that doesn't need to be secure bases they used for their measurements and divide the photons received into two groups: the first consists of photons measured in the same basis by Alice and Bob, and the second contains photons measured in different bases. 

The first group makes use of properties of entanglement. For each measurement Alice and Bob made using the same basis, there exists this perfect anti-correlation between Alice and Bob's photon. This means that they end up each having a bit string which is the binary complement of each other. Then Bob can just invert his bits and obtain a shared secret key with Alice to use for one-time pad. 

The second group is used to detect eavesdropping, they can compute the test statistic S which consists of expected values of Alice and Bob's measurements on the basis of their own choice \footnote{We consider the result "0" as having value 1 (spin up), and the result "1" as having value -1 (spin down). See \cite{Ekert1991} for more details on how the test statistic S is computed.}. It turns out that without eavesdropping, $S=-2\sqrt{2}$. Yet if Eve tapped into the communication, then $-\sqrt{2} \le S \le \sqrt{2}$. Such way of statistically testing for eavesdropping originated from the generalized Bell's theorem (CHSH inequalities) \cite{bell1964}, the concept of which was illustrated in \textbf{Section \ref{section: bell-nonlocality}}.

% This protocol is based on the Bohm's version of the Einsteain-Podolsky-Rosen gedanken experiment, and the generalized Bell's theorem (CHSH inequalities) is used to test for eavesdropping. From a theoretical points of view, E91 protocol extends BB84. 
