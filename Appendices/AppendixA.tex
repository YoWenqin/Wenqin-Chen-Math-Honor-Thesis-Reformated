% Appendix A

\chapter{Four Postulates of Quantum Mechanics With Density Matrices} % Main appendix title

\label{AppendixA} % For referencing this appendix elsewhere, use \ref{AppendixA}

\textbf{Postulate 1}: Associated to any isolated physical system is a complex vector space with inner product (that is, a Hilbert space) known as the \textit{state space} of the system. The system is completely described by its \textit{density matrix}, which is a positive matrix with trace one, acting on the state space of the system. If a quantum system is in the state $\rho_i$ with probability $p_i$, then the density matrix for the system is $\sum_i p_i \rho_i$

\textbf{Postulate 2}: The evolution of a closed quantum system is described by a \textit{unitary transformation}. That is, the state $\rho$ of the system at time $t_1$ is related to the state $\rho'$ of the system at time $t_2$ by a unitary operator U which depends only on the times $t_1$ and $t_2$, $\rho'=U\rho U^\dagger$.

\textbf{Postulate 3}: Quantum measurements are described by a collection $\{M_m\}$ of \textit{measurement operators}. These are operators acting on the state space of the system being measured. The index m refers to the measurement outcomes that may occur in the experiment. If the state of the quantum system is $\ket{\psi}$ immediately before the measurement then the probability that result m occurs is given by 
    \begin{equation}
        p(m)=tr(M_m^\dagger M_m \rho)
    \end{equation}
    and the state of the system after the measurement is
    \begin{equation}
        \frac{M_m \rho M_m^\dagger}{tr(M_m^\dagger M_m \rho)}
    \end{equation}
    The measurement operators satisfy the completeness equation $\sum_m M_m^\dagger M_m =I$
    
\textbf{Postulate 4}:The state space of a composite physical system is the tensor product of the component physical systems. Moreover, if we have systems numbered 1 through n, and system number i is prepared in the state $\rho_i$, then the joint state of the total system is $\rho_1 \otimes \ket{\psi_1} \otimes \hdots \otimes \rho_n$

