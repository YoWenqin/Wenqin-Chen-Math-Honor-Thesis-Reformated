% Appendix A

\chapter{Projective Measurements} % Main appendix title

\label{AppendixA} % Change X to a consecutive letter; for referencing this appendix elsewhere, use \ref{AppendixX}
In this appendix, we include some more general notions of measurement than measuring with respect to an orthonormal basis.  First, {\bf{Postulate 3}} can be completely restated in the context of projective measurements.  We present this here.
\begin{quote}
    \textbf{Projective Measurements}: A projective measurement is described by an \textit{observable}, M, a Hermitian operator on the state space of the sytem being observed. The observable has a spectral decomposition,
    \begin{equation}
        M=\sum_m m P_m
    \end{equation}
    where $P_m$ is the projector onto the eigenspace of M with eigenvalue m \footnote{The eigenspace corresponding to an eigenvalue m is the set of vectors which have eigenvalue m}. The possible outcomes of the measurement correspond to the eigenvalues, m, of the observable. Upon measuring the state $\ket{\psi}$, the probability of getting result m is given by 
    \begin{equation} \label{eqn: projective measurement}
        p(m)=\braket{\psi|P_m|\psi}
    \end{equation}
    Given that outcome m occurred, the state of the quantum system immediately after the measurement is 
    \begin{equation}
        \frac{P_m \ket{\psi}}{\sqrt{p(m)}}.
    \end{equation}
\end{quote}

Here is an example of projective measurements on single qubits.
\begin{example}
Consider the Pauli-Z matrix introduced in \textbf{Example} \ref{example-pauli z diagonal rep}.
\begin{equation}
    Z=\ket{0}\bra{0}-\ket{1}\bra{1}
\end{equation}
Take Z as the observable for out projective measurement. Here $P_{+1}=\ket{0}\bra{0}$ is the projector onto the eigenspace of Z with eigenvalue 1. $P_{-1}=\ket{1}\bra{1}$ is the projector onto the eigenspace of Z with eigenvalue -1.

Thus the measurement of Z on the state $\ket{\psi}=\frac{\ket{0}+\ket{1}}{\sqrt{2}}$ yields the outcome 1 with probability $\braket{\psi|0}\braket{0|\psi}=\frac{1}{2}$. Similarly, the outcome -1 has probability $\frac{1}{2}$.
\end{example}

% Another feature that distinguishes quantum from classical is that that we can measure a qubit in basis aside from the standard basis. 


Lastly, we point out that a nice property of projective measurements is that it is very easy to calculate the expected value of a given the observable M of a projective measurement:
\begin{eqnarray}
E(M)&=&\sum_m mp(m)\\
&=&\sum_m m \braket{\psi|P_m|\psi}\\
&=&\braket{\psi|(\sum_m mP_m)|\psi}\\
&=&\braket{\psi|M|\psi}.
\end{eqnarray}

